\documentclass[12pt]{article}
\usepackage[utf8]{inputenc}
\usepackage{geometry}
\usepackage{longtable}
\geometry{a4paper, margin=1in}

\begin{document}


\begin{center}
    \Large \textbf{SFWRENG 4G06A} \\[10pt]
    \normalsize Team 24, UNO flip
\end{center}


\begin{center}
    
        \item Mingyang Xu
        \item Kevin Ishak
        \item Zheng Bang Liang
        \item Zain-Alabedeen Garada
        \item Jianhao Wei
    
\end{center}

This document summarizes the contributions of each team member up to the POC Demo. The time period of interest is the time between the beginning of the term and the POC demo.


\section{Demo Plan: Showcasing Real-Time Client-Server Synchronization in a Multiplayer Game}

In this demo, we will present the client and server architecture of the multiplayer game we have developed, highlighting the seamless synchronization of game state and player interactions in real-time. The demonstration will emphasize the following key aspects:

\subsection{Core Functionality Overview}
We will begin by explaining the foundational architecture of the game, focusing on how the client and server communicate to ensure a consistent game state across all connected players. This includes showcasing the role of the server as the authoritative source of truth and the clients as interfaces for player actions.

\subsection{Real-Time Synchronization}
The demo will illustrate the real-time exchange of data between the client and server. We will demonstrate how player actions, such as movement, object interactions, or other in-game events, are captured by the client and synchronized with the server. The server then processes these events and updates all connected clients to maintain a unified game experience.

\subsection{Key Features in Action}
\begin{itemize}
    \item \textbf{State Consistency:} We'll demonstrate how the server handles multiple clients to maintain consistency in the game state, even under conditions like simultaneous actions or conflicting inputs.
    \item \textbf{Latency Handling:} A brief insight into how our game minimizes latency effects, ensuring a smooth user experience.
    \item \textbf{Scalability:} We'll explain how the architecture supports scalability, enabling multiple clients to join and participate in the game without compromising performance.
\end{itemize}

\subsection{Live Interaction}
To conclude, we will run a live demonstration of the game, showing two or more clients interacting in real-time. Attendees will see how actions performed by one player are instantly reflected in the game environment of other players, illustrating the robustness and reliability of our synchronization mechanism.


\section{Team Meeting Attendance}
\textit{[For each team member how many team meetings have they attended over the time period of interest. This number should be determined from the meeting issues in the team’s repo. The first entry in the table should be the total number of team meetings held by the team. ---SS]}

\begin{longtable}{|l|c|}
\hline
\textbf{Student} & \textbf{Team's Meetings} \\
\hline
total & 10 \\
Mingyang Xu & 9 \\
Kevin Ishak & 8 \\
Zheng Bang Liang & 8 \\
Zain-Alabedeen Garada & 9 \\
Jianhao Wei & 9 \\
\hline
\end{longtable}

\textit{[If needed, an explanation for the counts can be provided here. ---SS]}

\section{Supervisor/Stakeholder Meeting Attendance}
\textit{[For each team member how many supervisor/stakeholder team meetings have they attended over the time period of interest. This number should be determined from the supervisor meeting issues in the team’s repo. ---SS]}

\begin{longtable}{|l|c|}
\hline
\textbf{Student} & \textbf{Supervisor Meetings} \\
\hline
total & 1 \\
Mingyang Xu & 1 \\
Kevin Ishak & 1 \\
Zheng Bang Liang & 1 \\
Zain-Alabedeen Garada & 1 \\
Jianhao Wei & 1 \\
\hline
\end{longtable}

\textit{[If needed, an explanation for the counts can be provided here. ---SS]}

\section{Lecture Attendance}
\textit{[For each team member how many lectures have they attended over the time period of interest. This number should be determined from the lecture issues in the team’s repo. ---SS]}

\begin{longtable}{|l|c|}
\hline
\textbf{Student} & \textbf{Lectures} \\
\hline
total & 11 \\
Mingyang Xu & 8 \\
Kevin Ishak & 4 \\
Zheng Bang Liang & 5 \\
Zain-Alabedeen Garada & 4 \\
Jianhao Wei & 7 \\
\hline
\end{longtable}

\textit{[If needed, an explanation for the lecture attendance can be provided here. ---SS]}

\section{TA Document Discussion Attendance}
\textit{[For each team member how many of the informal document discussion meetings with the TA were attended over the time period of interest. ---SS]}

\begin{longtable}{|l|c|}
\hline
\textbf{Student} & \textbf{TA meeting} \\
\hline
total & 3 \\
Mingyang Xu & 3 \\
Kevin Ishak & 3 \\
Zheng Bang Liang & 2 \\
Zain-Alabedeen Garada & 3 \\
Jianhao Wei & 2 \\
\hline
\end{longtable}

\textit{[If needed, an explanation for the attendance can be provided here. ---SS]}

\section{Commits}
\textit{[For each team member how many commits to the main branch have been made over the time period of interest. ---SS]}

\begin{longtable}{|l|c|c|}
\hline
\textbf{Student} & \textbf{Commits} & \textbf{Percent} \\
\hline
Total & 31 & 100\% \\
Mingyang Xu & 15 & 48\% \\
Jianhao Wei & 10 & 32\% \\
Zain-Alabedeen Garada & 6 & 19\% \\
Kevin Ishak & 6 & 19\% \\
Andy Liang & 0 & 0\% \\
\hline
\end{longtable}

\textit{[If needed, an explanation for the counts can be provided here. ---SS]}
\\Note: The commit number does not necessarily represent how much effort each group member put into as most of our documents are done on Google Docs and uploaded by a single person.
For implementation some work was done together using Live Share on VS Code but committed by 1 person.

\section{Issue Tracker}
\textit{[For each team member how many issues have they authored (including open and closed issues (O+C)) and how many have they been assigned (only counting closed issues (C only)) over the time period of interest. ---SS]}

\begin{longtable}{|l|c|c|}
\hline
\textbf{Student} & \textbf{Authored (O+C)} & \textbf{Assigned (C only)} \\
\hline
Jianhao Wei & 0 & 0 \\
Mingyang Xu & 4 & 0 \\
Zain-Alabedeen Garada & 6 & 0 \\
Kevin Ishak & 0 & 0 \\
Andy Liang & 0 & 0 \\
\hline
\end{longtable}

\textit{[If needed, an explanation for the counts can be provided here. ---SS]}
\\Although Jianhao Wei did not author any issues by himself, he helped Zain-Alabedeen Garada to check over and make correction with the issue he authored

\section{CICD}
\textit{[Say how CICD will be used in your project ---SS]}

\textit{[If your team has additional metrics of productivity, please feel free to add them to this report. ---SS]}

\end{document}
