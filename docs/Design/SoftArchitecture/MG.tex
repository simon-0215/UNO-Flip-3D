\documentclass[12pt, titlepage]{article}

\usepackage{fullpage}
\usepackage[round]{natbib}
\usepackage{multirow}
\usepackage{booktabs}
\usepackage{tabularx}
\usepackage{graphicx}
\usepackage{float}
\usepackage{hyperref}
\hypersetup{
    colorlinks,
    citecolor=blue,
    filecolor=black,
    linkcolor=red,
    urlcolor=blue
}


\newcounter{acnum}
\newcommand{\actheacnum}{AC\theacnum}
\newcommand{\acref}[1]{AC\ref{#1}}

\newcounter{ucnum}
\newcommand{\uctheucnum}{UC\theucnum}
\newcommand{\uref}[1]{UC\ref{#1}}

\newcounter{mnum}
\newcommand{\mthemnum}{M\themnum}
\newcommand{\mref}[1]{M\ref{#1}}

\begin{document}

\title{Module Guide for \progname{}} 
\author{\authname}
\date{\today}

\maketitle

\pagenumbering{roman}

\section{Revision History}

\begin{tabularx}{\textwidth}{p{3cm}p{2cm}X}
\toprule {\bf Date} & {\bf Version} & {\bf Notes}\\
\midrule
2025 Jan 13 & 1.0 & Start section 3-5\\
2025 Jan 14 & 1.1 & Start section 5-7\\
\bottomrule
\end{tabularx}

\newpage

\section{Reference Material}

This section records information for easy reference.

\subsection{Abbreviations and Acronyms}

\renewcommand{\arraystretch}{1.2}
\begin{tabular}{l l} 
  \toprule		
  \textbf{symbol} & \textbf{description}\\
  \midrule 
  AC & Anticipated Change\\
  DAG & Directed Acyclic Graph \\
  M & Module \\
  MG & Module Guide \\
  OS & Operating System \\
  R & Requirement\\
  SC & Scientific Computing \\
  SRS & Software Requirements Specification\\
  \progname & Explanation of program name\\
  UC & Unlikely Change \\
  \wss{etc.} & \wss{...}\\
  \bottomrule
\end{tabular}\\

\newpage

\tableofcontents

\listoftables

\listoffigures

\newpage

\pagenumbering{arabic}

\section{Introduction}

Developing software through modular decomposition is a well-accepted practice in software engineering. A module is defined as a work assignment for a programmer or programming team 24. This approach allows the software system to be divided into manageable components, enhancing scalability, maintainability, and adaptability. In this project, we follow the principle of information hiding~\citep{Parnas1972a}, which is essential for accommodating changes during development and maintenance. Given the dynamic and interactive nature of the UNO Flip game, this principle is particularly relevant to ensure flexibility as new features or rules may be incorporated in the future.

UNO Flip is a modern twist on the traditional UNO card game, incorporating an innovative double-sided card deck with "light" and "dark" sides. Players are challenged to adapt their strategies dynamically as the game flips between these two modes. Our goal for this project is to design and develop a digital version of UNO Flip that emulates the physical gameplay experience while adding features like automated rule enforcement, multiplayer support, and interactive animations.

To ensure a robust and maintainable design, we adhere to the modular design principles established by \citet{ParnasEtAl1984}:

\begin{itemize}
\item System details that are likely to change independently, such as game rules or graphical user interface (GUI) elements, are encapsulated within separate modules.
\item Each data structure, such as the card deck, player hands, or game state, is implemented in only one module.
\item Inter-module communication is facilitated through well-defined access programs to ensure data encapsulation and minimize dependencies.
\end{itemize}

After completing the initial design phase, including the Software Requirements Specification (SRS), we created the Module Guide (MG)~\citep{ParnasEtAl1984}. The MG outlines the modular structure of the system and serves as a comprehensive reference for all stakeholders. This document is aimed at the following audiences:

\begin{itemize}
\item \textbf{New project members:} The MG provides a clear overview of the system's modular structure, enabling new members to understand the architecture and locate relevant modules efficiently.
\item \textbf{Maintainers:} The hierarchical organization of the MG simplifies the process of identifying and updating the relevant modules when changes are made to the system. Maintainers are encouraged to update the MG to reflect any modifications accurately.
\item \textbf{Designers:} The MG acts as a verification tool to ensure consistency, feasibility, and flexibility in the system design. Designers can assess the coherence among modules, the viability of the decomposition, and the adaptability of the design to future changes.
\end{itemize}

The rest of this document is structured as follows. Section~\ref{SecChange} outlines the anticipated and unlikely changes in the software requirements, such as the introduction of new game rules or enhancements to the multiplayer functionality. Section~\ref{SecMH} describes the module decomposition strategy, constructed based on the principle of likely changes. Section~\ref{SecConnection} details the connections between the software requirements and the modules. Section~\ref{SecMD} provides a comprehensive description of each module, including its responsibilities and dependencies. Section~\ref{SecTM} includes traceability matrices to ensure the completeness of the design relative to the requirements and anticipated changes. Finally, Section~\ref{SecUse} discusses the use relationships between modules, highlighting their interactions and dependencies.

This modular approach ensures that our UNO Flip software can be developed efficiently while maintaining the flexibility to adapt to future requirements and user feedback.


\section{Anticipated and Unlikely Changes} \label{SecChange}

This section lists possible changes to the system. According to the likeliness
of the change, the possible changes are classified into two
categories. Anticipated changes are listed in Section \ref{SecAchange}, and
unlikely changes are listed in Section \ref{SecUchange}.

\subsection{Anticipated Changes} \label{SecAchange}

Anticipated changes are the source of the information that is to be hidden
inside the modules. Ideally, changing one of the anticipated changes will only
require changing the one module that hides the associated decision. The approach
adapted here is called design for
change.

\begin{description}
\item[\refstepcounter{acnum} \actheacnum \label{acHardware}:] The specific
  hardware on which the software is running. \\
  \textit{Explanation:} The game should adapt to various hardware platforms, such as desktops, mobile devices, and consoles, by leveraging Unity's cross-platform build tools.
  
\item[\refstepcounter{acnum} \actheacnum \label{acInput}:] The format of the
  initial input data. \\
  \textit{Explanation:} Unity's Input System allows for flexibility in handling various input methods, such as touch, gamepad, or keyboard input, without impacting the overall game functionality.
  
\item[\refstepcounter{acnum} \actheacnum \label{acUI}:] Updating the user interface for accessibility and improved user experience. \\
  \textit{Explanation:} Changes in UI elements, such as adding new accessibility features or redesigning layouts, can be achieved without affecting the underlying game logic due to Unity's modular UI Toolkit.

\item[\refstepcounter{acnum} \actheacnum \label{acNetwork}:] The communication protocol for multiplayer functionality. \\
  \textit{Explanation:} Switching from an existing protocol (e.g., UDP) to a more secure or efficient one (e.g., WebSockets) should only require modifications to the networking module.
\end{description}

\wss{Anticipated changes relate to changes that would be made in requirements,
design or implementation choices.  They are not related to changes that are made
at run-time, like the values of parameters.}

\subsection{Unlikely Changes} \label{SecUchange}

The module design should be as general as possible. However, a general system is
more complex. Sometimes this complexity is not necessary. Fixing some design
decisions at the system architecture stage can simplify the software design. If
these decisions should later need to be changed, then many parts of the design
will potentially need to be modified. Hence, it is not intended that these
decisions will be changed.

\begin{description}
\item[\refstepcounter{ucnum} \uctheucnum \label{ucIO}:] Input/Output devices
  (Input: File and/or Keyboard, Output: File, Memory, and/or Screen). \\
  \textit{Explanation:} Changing I/O devices is considered unlikely since the Unity Input System already supports a wide range of devices, and most use cases are covered by the current implementation.

\item[\refstepcounter{ucnum} \uctheucnum \label{ucEngine}:] Switching the game engine from Unity to another platform. \\
  \textit{Explanation:} Rebuilding the game using a different engine would require re-implementation of all assets, scripts, and logic, making this change highly unlikely.

\item[\refstepcounter{ucnum} \uctheucnum \label{ucMechanics}:] Fundamental changes to the core gameplay mechanics. \\
  \textit{Explanation:} Altering the basic rules of UNO Flip, such as removing the "Flip" mechanic, would require significant rewrites across multiple modules.

\item[\refstepcounter{ucnum} \uctheucnum \label{ucAssets}:] Replacing all graphical assets with a new visual theme. \\
  \textit{Explanation:} Although possible, replacing all assets would involve modifying Unity scenes, prefabs, and animations extensively, which is not a likely requirement.
\end{description}


\section{Module Hierarchy} \label{SecMH}

This section provides an overview of the module design. Modules are summarized
in a hierarchy decomposed by secrets in Table \ref{TblMH}. The modules listed
below, which are leaves in the hierarchy tree, are the modules that will
actually be implemented.

\begin{description}
\item [\refstepcounter{mnum} \mthemnum \label{mHH}:] Hardware-Hiding Module \\
  \textit{Purpose:} This module isolates hardware-specific dependencies, such as rendering devices, input devices, and memory management. It allows the rest of the system to remain platform-independent by using Unity's built-in hardware abstraction.
  
\item [\refstepcounter{mnum} \mthemnum \label{mBH}:] Behaviour-Hiding Module \\
  \textit{Purpose:} Encapsulates the core gameplay logic, such as player actions, turn mechanics, and rule enforcement, ensuring that other modules can interact with game behavior through a unified interface.

\item [\refstepcounter{mnum} \mthemnum \label{mSD}:] Software Decision Module \\
  \textit{Purpose:} Handles system-wide decisions, including UI updates, multiplayer network protocols, and game state transitions. This module integrates decisions that impact user interaction and system-wide consistency.

\end{description}


\begin{table}[h!]
\centering
\begin{tabular}{p{0.3\textwidth} p{0.6\textwidth}}
\toprule
\textbf{Level 1} & \textbf{Level 2}\\
\midrule

{Hardware-Hiding Module} & \textit{Hardware Abstraction module: Isolate hardware-specific dependencies, such as rendering devices and input/output handling.}\\
\midrule

\multirow{5}{0.3\textwidth}{Behaviour-Hiding Module} 
& Game Logic Module: Handles player actions, turn mechanics, and card rules enforcement.\\
& Turn Management Module: Ensures correct sequencing of turns, including handling "Skip" or "Reverse" cards.\\
& Card Effect Module: Implements special card effects such as "Flip" or "Draw Two."\\
& Score Tracking Module: Manages player scores and determines game-winning conditions.\\
& Animation Module: Handles animations for card movements, flips, and effects using Unity's animation system.\\

\midrule

\multirow{3}{0.3\textwidth}{Software Decision Module} 
& UI Module: Manages user interface components, including menus, HUD, and accessibility options.\\
& Multiplayer Networking Module: Handles online player interactions, matchmaking, and game state synchronization.\\
& Save/Load Module: Provides functionality for saving and loading game progress.\\

\bottomrule

\end{tabular}
\caption{Module Hierarchy}
\label{TblMH}
\end{table}


\section{Connection Between Requirements and Design} \label{SecConnection}

The design of the system is intended to satisfy the requirements developed in
the SRS. In this stage, the system is decomposed into modules. The connection
between requirements and modules is listed in Table~\ref{TblRT}.

\wss{The intention of this section is to document decisions that are made
  ``between'' the requirements and the design.  To satisfy some requirements,
  design decisions need to be made.  Rather than make these decisions implicit,
  they are explicitly recorded here.  For instance, if a program has security
  requirements, a specific design decision may be made to satisfy those
  requirements with a password.}

\section{Module Decomposition} \label{SecMD}

Modules are decomposed according to the principle of ``information hiding''
proposed by \citet{ParnasEtAl1984}. The \emph{Secrets} field in a module
decomposition is a brief statement of the design decision hidden by the
module. The \emph{Services} field specifies \emph{what} the module will do
without documenting \emph{how} to do it. For each module, a suggestion for the
implementing software is given under the \emph{Implemented By} title. If the
entry is \emph{OS}, this means that the module is provided by the operating
system or by standard programming language libraries.  \emph{\progname{}} means the
module will be implemented by the \progname{} software.

Only the leaf modules in the hierarchy have to be implemented. If a dash
(\emph{--}) is shown, this means that the module is not a leaf and will not have
to be implemented.

\subsection{Hardware Hiding Modules (\mref{mHH})}

\begin{description}
\item[Secrets:] The data structure and algorithm used to implement the virtual
  hardware.
\item[Services:] Serves as a virtual hardware used by the rest of the
  system. This module provides the interface between the hardware and the
  software, allowing the system to display outputs or accept inputs.
\item[Implemented By:] OS
\end{description}

\subsection{Behaviour-Hiding Module}

\begin{description}
\item[Secrets:] The contents of the required behaviours, including turn management, game logic, and card effects.
\item[Services:] Includes programs that provide externally visible behaviours of
  the system as specified in the software requirements specification (SRS)
  documents. This module serves as a communication layer between the
  hardware-hiding module and the software decision module. The programs in this
  module will need to change if there are changes in the SRS.
\item[Implemented By:] --
\end{description}

\subsubsection{Input Format Module (\mref{mInput})}

\begin{description}
\item[Secrets:] The format and structure of the input data, including data serialization for communication.
\item[Services:] Converts the input data into the data structure used by other
  modules, such as the game logic or UI modules.
\item[Implemented By:] \progname{}
\item[Type of Module:] Abstract Data Type
\end{description}

\subsubsection{Card Effect Module (\mref{mCardEffect})}

\begin{description}
\item[Secrets:] The implementation details of special card effects, such as "Flip," "Skip," and "Draw Two."
\item[Services:] Executes the effects of special cards and updates the game state accordingly.
\item[Implemented By:] \progname{}
\item[Type of Module:] Abstract Object
\end{description}

\subsubsection{Turn Management Module (\mref{mTurn})}

\begin{description}
\item[Secrets:] The sequence and rules for determining the current player.
\item[Services:] Manages the order of player turns, including handling special conditions like "Reverse" or "Skip" cards.
\item[Implemented By:] \progname{}
\item[Type of Module:] Record
\end{description}

\subsection{Software Decision Module}

\begin{description}
\item[Secrets:] The design decisions based on performance optimizations, networking protocols, and user interaction considerations. These secrets are \emph{not} described in the SRS.
\item[Services:] Includes data structures and algorithms used in the system that
  do not provide direct interaction with the user, such as multiplayer matchmaking and game state synchronization.
\item[Implemented By:] \progname{}
\end{description}

\subsubsection{Multiplayer Networking Module (\mref{mNetwork})}

\begin{description}
\item[Secrets:] The implementation details of real-time communication, including protocols like UDP or WebSocket.
\item[Services:] Handles communication between players, including matchmaking, game state synchronization, and latency management.
\item[Implemented By:] \progname{}
\item[Type of Module:] Library
\end{description}

\subsubsection{UI Module (\mref{mUI})}

\begin{description}
\item[Secrets:] The layout and design of user interface components, such as menus, HUD, and accessibility features.
\item[Services:] Displays the game state to the user and accepts user inputs through various interactive elements.
\item[Implemented By:] \progname{}
\item[Type of Module:] Abstract Object
\end{description}


\section{Traceability Matrix} \label{SecTM}

This section shows two traceability matrices: between the modules and the
requirements and between the modules and the anticipated changes.

% the table should use mref, the requirements should be named, use something
% like fref
\begin{table}[H]
\centering
\begin{tabular}{p{0.2\textwidth} p{0.6\textwidth}}
\toprule
\textbf{Req.} & \textbf{Modules}\\
\midrule
R1 & \mref{mHH}, \mref{mInput}, \mref{mParams}, \mref{mControl}\\
R2 & \mref{mInput}, \mref{mParams}\\
R3 & \mref{mVerify}\\
R4 & \mref{mOutput}, \mref{mControl}\\
R5 & \mref{mOutput}, \mref{mODEs}, \mref{mControl}, \mref{mSeqDS}, \mref{mSolver}, \mref{mPlot}\\
R6 & \mref{mOutput}, \mref{mODEs}, \mref{mControl}, \mref{mSeqDS}, \mref{mSolver}, \mref{mPlot}\\
R7 & \mref{mOutput}, \mref{mEnergy}, \mref{mControl}, \mref{mSeqDS}, \mref{mPlot}\\
R8 & \mref{mOutput}, \mref{mEnergy}, \mref{mControl}, \mref{mSeqDS}, \mref{mPlot}\\
R9 & \mref{mVerifyOut}\\
R10 & \mref{mOutput}, \mref{mODEs}, \mref{mControl}\\
R11 & \mref{mOutput}, \mref{mODEs}, \mref{mEnergy}, \mref{mControl}\\
\bottomrule
\end{tabular}
\caption{Trace Between Requirements and Modules}
\label{TblRT}
\end{table}

\begin{table}[H]
\centering
\begin{tabular}{p{0.2\textwidth} p{0.6\textwidth}}
\toprule
\textbf{AC} & \textbf{Modules}\\
\midrule
\acref{acHardware} & \mref{mHH}\\
\acref{acInput} & \mref{mInput}\\
\acref{acParams} & \mref{mParams}\\
\acref{acVerify} & \mref{mVerify}\\
\acref{acOutput} & \mref{mOutput}\\
\acref{acVerifyOut} & \mref{mVerifyOut}\\
\acref{acODEs} & \mref{mODEs}\\
\acref{acEnergy} & \mref{mEnergy}\\
\acref{acControl} & \mref{mControl}\\
\acref{acSeqDS} & \mref{mSeqDS}\\
\acref{acSolver} & \mref{mSolver}\\
\acref{acPlot} & \mref{mPlot}\\
\bottomrule
\end{tabular}
\caption{Trace Between Anticipated Changes and Modules}
\label{TblACT}
\end{table}

\section{Use Hierarchy Between Modules} \label{SecUse}

In this section, the uses hierarchy between modules is
provided. \citet{Parnas1978} said of two programs A and B that A {\em uses} B if
correct execution of B may be necessary for A to complete the task described in
its specification. That is, A {\em uses} B if there exist situations in which
the correct functioning of A depends upon the availability of a correct
implementation of B.  Figure \ref{FigUH} illustrates the use relation between
the modules. It can be seen that the graph is a directed acyclic graph
(DAG). Each level of the hierarchy offers a testable and usable subset of the
system, and modules in the higher level of the hierarchy are essentially simpler
because they use modules from the lower levels.

\wss{The uses relation is not a data flow diagram.  In the code there will often
be an import statement in module A when it directly uses module B.  Module B
provides the services that module A needs.  The code for module A needs to be
able to see these services (hence the import statement).  Since the uses
relation is transitive, there is a use relation without an import, but the
arrows in the diagram typically correspond to the presence of import statement.}

\wss{If module A uses module B, the arrow is directed from A to B.}

\begin{figure}[H]
\centering
%\includegraphics[width=0.7\textwidth]{UsesHierarchy.png}
\caption{Use hierarchy among modules}
\label{FigUH}
\end{figure}

%\section*{References}

\section{User Interfaces}

\wss{Design of user interface for software and hardware.  Attach an appendix if
needed. Drawings, Sketches, Figma}

\section{Design of Communication Protocols}

\wss{If appropriate}

\section{Timeline}

\wss{Schedule of tasks and who is responsible}

\wss{You can point to GitHub if this information is included there}

\bibliographystyle {plainnat}
\bibliography{../../../refs/References}

\newpage{}

\end{document}
