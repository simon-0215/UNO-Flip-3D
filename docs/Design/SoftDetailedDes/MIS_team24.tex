\documentclass[12pt, titlepage]{article}

\usepackage{amsmath, mathtools}

\usepackage[round]{natbib}
\usepackage{amsfonts}
\usepackage{amssymb}
\usepackage{graphicx}
\usepackage{colortbl}
\usepackage{xr}
\usepackage{hyperref}
\usepackage{longtable}
\usepackage{xfrac}
\usepackage{tabularx}
\usepackage{float}
\usepackage{siunitx}
\usepackage{booktabs}
\usepackage{multirow}
\usepackage[section]{placeins}
\usepackage{caption}
\usepackage{fullpage}

\hypersetup{
bookmarks=true,     % show bookmarks bar?
colorlinks=true,       % false: boxed links; true: colored links
linkcolor=red,          % color of internal links (change box color with linkbordercolor)
citecolor=blue,      % color of links to bibliography
filecolor=magenta,  % color of file links
urlcolor=cyan          % color of external links
}

\usepackage{array}

\externaldocument{../../SRS/SRS}



\begin{document}

\title{Module Interface Specification for \progname{UnoFlip3D}}

\author{\authname Mingyang Xu \\ Kevin Ishak \\ Jianhao Wei \\ Zain-Alabedeen Garada \\ Zheng Beng Liang}

\date{\today}

\maketitle

\pagenumbering{roman}

\section{Revision History}

\begin{tabularx}{\textwidth}{p{4cm}p{3cm}X}
\toprule {\bf Date} & {\bf Developer} & {\bf Notes}\\
\midrule
January 12th, 2025 & Kevin Ishak & Initialize template, add rough draft of section 3,4 and 5\\
January 13th, 2025 & Jianhao Wei & Add modules into section 6\\
January 13th, 2025 & Zain-Alabedeen Garada & Add modules into section 6\\
January 17th, 2025 & Zheng Bang Liang & Added all modules for Behavior Hiding Modules\\
\bottomrule
\end{tabularx}

~\newpage

\section{Symbols, Abbreviations and Acronyms}

See SRS Documentation at \wss{give url}

\wss{Also add any additional symbols, abbreviations or acronyms}

\newpage

\tableofcontents

\newpage

\pagenumbering{arabic}

\section{Introduction}

UNO Flip is a modern twist on the traditional UNO card game, incorporating an innovative double-sided card deck with ”light” and ”dark” sides. Players are challenged to adapt their strategies dynamically as the game flips between these two modes. Our goal for this project is to design and develop a digital version of UNO Flip that emulates the physical gameplay experience while adding features like automated rule enforcement, multiplayer support, and interactive animations.

Complementary documents include the System Requirement Specifications and Module Guide. The full documentation and implementation can be found at \url{https://github.com/simon-0215/UNO-Flip-3D/tree/main} .


\section{Notation}

\wss{You should describe your notation.  You can use what is below as
  a starting point.}

The structure of the MIS for modules comes from \citet{HoffmanAndStrooper1995},
with the addition that template modules have been adapted from
\cite{GhezziEtAl2003}.  The mathematical notation comes from Chapter 3 of
\citet{HoffmanAndStrooper1995}.  For instance, the symbol := is used for a
multiple assignment statement and conditional rules follow the form $(c_1
\Rightarrow r_1 | c_2 \Rightarrow r_2 | ... | c_n \Rightarrow r_n )$.

The following table summarizes the primitive data types used by \progname. 

\begin{center}
\renewcommand{\arraystretch}{1.2}
\noindent 
\begin{tabular}{l l p{7.5cm}} 
\toprule 
\textbf{Data Type} & \textbf{Notation} & \textbf{Description}\\ 
\midrule
character & char & a single symbol or digit\\
integer & $\mathbb{Z}$ & a number without a fractional component in (-$\infty$, $\infty$) \\
natural number & $\mathbb{N}$ & a number without a fractional component in [1, $\infty$) \\
real & $\mathbb{R}$ & any number in (-$\infty$, $\infty$)\\
\bottomrule
\end{tabular} 
\end{center}

\noindent
The specification of \progname \ uses some derived data types: sequences, strings, and
tuples. Sequences are lists filled with elements of the same data type. Strings
are sequences of characters. Tuples contain a list of values, potentially of
different types. In addition, \progname \ uses functions, which
are defined by the data types of their inputs and outputs. Local functions are
described by giving their type signature followed by their specification.

\section{Module Decomposition}

The following table is taken directly from the Module Guide document for this project.

\begin{table}[h!]
\centering
\begin{tabular}{p{0.3\textwidth} p{0.6\textwidth}}
\toprule
\textbf{Level 1} & \textbf{Level 2}\\
\midrule

\multirow{3}{0.3\textwidth}{Hardware-Hiding} & Device Input\\
& Audio Output\\
& Screen Rendering\\
\midrule

\multirow{4}{0.3\textwidth}{Behaviour-Hiding} & Game Rules Logic\\
& Turn Management\\
& Card Management\\
& Player Interaction\\
\midrule

\multirow{2}{0.3\textwidth}{Software Decision} 
& Networking Protocol\\
& Multiplayer System\\
\bottomrule

\end{tabular}
\caption{Module Hierarchy}
\label{TblMH}
\end{table}

\newpage
~\newpage


%Section 6

\section{MIS of Turn Management Module} 

\subsection{Module}
\hspace{1.5em}Turn Management

\subsection{Uses}
\hspace{1.5em}Multiplayer, UI, AssetManagement, Server

\subsection{Syntax}

\subsubsection{Exported Constants}
\hspace{1.5em}None

\subsubsection{Exported Access Programs}

\begin{itemize}
\item validateMove(playerId, cardid)
\item endTurn(playerId)
\item shuffleDeck()
\item drawCard(playerId)
\end{itemize}

\subsection{Semantics}

\subsubsection{State Variables}
\begin{itemize}
\item currentPlayer: Tracks the player whose turn it is
\item deck: Represents the stack of remaining cards in the game.
\item discardPile: Stores played cards
\item playerHands: Stores each player's cards
\end{itemize}

\subsubsection{Environment Variables}
\begin{itemize}
\item maxPlayers: Maximum number of players allowed in a game
\item flipEnabled: Boolean to toggle the flip functionality
\end{itemize}

\subsubsection{Assumptions}
\begin{itemize}
\item The number of players, game rules, player restrictions are preloaded
\item The game environment is known
\end{itemize}

\subsubsection{Access Routine Semantics}

\begin{itemize}
\item validateMove(playerId, card)\\
Transition: Checks if a move is valid

\item endTurn(playerId)\\
Transition: Ends the current player's turn and starts the next

\item shuffleDeck()\\
Transition: Randomizes the card deck

\item drawCard(playerId)\\
Transition: Adds a card to the specified player’s hand

\end{itemize}


\subsubsection{Local Functions}
\begin{itemize}
\item shuffleProcess()\\
\textbf{Description:} Contain the random algorithm to shuffle the deck
\item CardModifier()\\
\textbf{Description:} Contain algorithm to draw different card to screen
\end{itemize}


%Section 7

\section{MIS of Multiplayer Module} 

\subsection{Module}
\hspace{1.5em}Multiplayer

\subsection{Uses}
\hspace{1.5em}GameLogic, Server

\subsection{Syntax}

\subsubsection{Exported Constants}
\hspace{1.5em}None

\subsubsection{Exported Access Programs}
\begin{itemize}
\item createGameRoom(playerId, roomSettings)
\item joinGameRoom(playerId, roomId)
\item broadcastUpdate(gameId, update)
\end{itemize}

\subsection{Semantics}

\subsubsection{State Variables}
\begin{itemize}
\item activeGames: Tracks all ongoing game sessions.
\item connectedPlayers: List of currently connected players.
\end{itemize}


\subsubsection{Environment Variables}
\begin{itemize}
\item serverIP: IP address of the game server.
\item timeoutLimit: Time limit for a player to respond during their turn.
\end{itemize}

\subsubsection{Assumptions}
\begin{itemize}
\item The connection to other machines can be established successfully
\item The encryption and decryption methods are known
\end{itemize}

\subsubsection{Access Routine Semantics}

\begin{itemize}
\item createGameRoom(playerId, roomSettings)\\
Creates a new game room


\item joinGameRoom(playerId, roomId)\\
Adds a player to an existing room

\item broadcastUpdate(gameId, update)\\
Sends game state updates to all players in a room.

\end{itemize}


\subsubsection{Local Functions}
\begin{itemize}
\item encryption(information)\\
\textbf{Description:} Contain encryption algorithm to encrypt data before sending
\item decryption(information)\\ 
\textbf{Description:} Contain decryption algorithm to decrypt data after receiving
\end{itemize}

\section{Behavior-Hiding Modules}

\subsection{Card Effect Module}

\subsubsection{Module}
Card Effect (CE)

\subsubsection{Uses}
Game Logic Module, Animation Module, Turn Management Module

\subsubsection{Syntax}

\paragraph{Exported Constants:}
\begin{itemize}
    \item \texttt{DRAW\_TWO\_EFFECT}: Identifier for the "Draw Two" card effect.
    \item \texttt{SKIP\_TURN\_EFFECT}: Identifier for the "Skip Turn" card effect.
    \item \texttt{FLIP\_DECK\_EFFECT}: Identifier for the "Flip Deck" card effect.
\end{itemize}

\paragraph{Exported Access Programs:}
\begin{itemize}
    \item \texttt{applyEffect(cardType: char, playerId: char) $\to$ void}
    \item \texttt{reverseDirection() $\to$ void}
    \item \texttt{skipTurn(playerId: char) $\to$ void}
    \item \texttt{triggerDrawCards(playerId: char, cardCount: \mathbb{N}) $\to$ void}
    \item \texttt{flipDeck() $\to$ void}
\end{itemize}

\subsubsection{Semantics}

\paragraph{State Variables:}
\begin{itemize}
    \item \texttt{currentEffect: char} -- Tracks the active card effect being applied.
    \item \texttt{effectQueue: sequence of char} -- Stores pending card effects to be applied sequentially.
\end{itemize}

\paragraph{Environment Variables:}
\begin{itemize}
    \item \texttt{specialCardTypes: sequence of char} -- List of predefined card types with special effects.
    \item \texttt{deckSide: char} -- Tracks the current state of the deck (light or dark).
\end{itemize}

\paragraph{Assumptions:}
\begin{itemize}
    \item All card effects are predefined and handled within this module.
    \item The "Flip" card effect toggles the deck state between light and dark.
\end{itemize}

\paragraph{Access Routine Semantics:}
\begin{itemize}
    \item \texttt{applyEffect(cardType: char, playerId: char):}
    \begin{itemize}
        \item \textbf{Transition:} Applies the specified card effect to the targeted player.
    \end{itemize}
    \item \texttt{reverseDirection():}
    \begin{itemize}
        \item \textbf{Transition:} Reverses the turn order of players.
    \end{itemize}
    \item \texttt{skipTurn(playerId: char):}
    \begin{itemize}
        \item \textbf{Transition:} Skips the specified player's turn.
    \end{itemize}
    \item \texttt{triggerDrawCards(playerId: char, cardCount: \mathbb{N}):}
    \begin{itemize}
        \item \textbf{Transition:} Forces the player to draw the specified number of cards.
    \end{itemize}
    \item \texttt{flipDeck():}
    \begin{itemize}
        \item \textbf{Transition:} Toggles the deck's state and updates the game logic accordingly.
    \end{itemize}
\end{itemize}

\paragraph{Local Functions:}
\begin{itemize}
    \item \texttt{calculateNextPlayer(direction: char) $\to$ char:}
    \begin{itemize}
        \item \textbf{Description:} Determines the next player based on the current direction.
    \end{itemize}
    \item \texttt{applyChainEffect(effectQueue: sequence of char) $\to$ void:}
    \begin{itemize}
        \item \textbf{Description:} Resolves multiple card effects in sequence, such as stacked "Draw Two" cards.
    \end{itemize}
\end{itemize}

\newpage

\subsection{Turn Management Module}

\subsubsection{Module}
Turn Management

\subsubsection{Uses}
Multiplayer Module, UI Module, Server Module

\subsubsection{Syntax}

\paragraph{Exported Constants:}
None

\paragraph{Exported Access Programs:}
\begin{itemize}
    \item \texttt{validateMove(playerId: char, card: char) $\to$ boolean}
    \item \texttt{endTurn(playerId: char) $\to$ void}
    \item \texttt{shuffleDeck() $\to$ void}
    \item \texttt{drawCard(playerId: char) $\to$ void}
\end{itemize}

\subsubsection{Semantics}

\paragraph{State Variables:}
\begin{itemize}
    \item \texttt{currentPlayer: char} -- Tracks the player whose turn it is.
    \item \texttt{deck: sequence of char} -- Represents the stack of cards remaining in the game.
    \item \texttt{discardPile: sequence of char} -- Stores cards that have been played.
    \item \texttt{playerHands: sequence of sequence of char} -- Stores the cards in each player's hand.
\end{itemize}

\paragraph{Environment Variables:}
\begin{itemize}
    \item \texttt{maxPlayers: \mathbb{N}} -- Maximum number of players allowed in the game.
    \item \texttt{flipEnabled: boolean} -- Boolean flag indicating if the "Flip" functionality is active.
\end{itemize}

\paragraph{Assumptions:}
\begin{itemize}
    \item The number of players and game rules are preloaded.
    \item Game environment variables are known and properly initialized.
\end{itemize}

\paragraph{Access Routine Semantics:}
\begin{itemize}
    \item \texttt{validateMove(playerId: char, card: char):}
    \begin{itemize}
        \item \textbf{Transition:} Verifies whether the move made by the player is valid.
    \end{itemize}
    \item \texttt{endTurn(playerId: char):}
    \begin{itemize}
        \item \textbf{Transition:} Ends the turn for the current player and initiates the next turn.
    \end{itemize}
    \item \texttt{shuffleDeck():}
    \begin{itemize}
        \item \textbf{Transition:} Randomizes the cards in the deck.
    \end{itemize}
    \item \texttt{drawCard(playerId: char):}
    \begin{itemize}
        \item \textbf{Transition:} Adds a card to the specified player's hand.
    \end{itemize}
\end{itemize}

\paragraph{Local Functions:}
\begin{itemize}
    \item \texttt{shuffleProcess() $\to$ void:}
    \begin{itemize}
        \item \textbf{Description:} Implements the randomization algorithm for shuffling the deck.
    \end{itemize}
    \item \texttt{updateTurnOrder() $\to$ void:}
    \begin{itemize}
        \item \textbf{Description:} Updates the order of players based on game logic.
    \end{itemize}
\end{itemize}

\subsection{User Interface Module}

\subsubsection{Module}
User Interface (UI)

\subsubsection{Uses}
Game Logic Module, Multiplayer Module, Asset Management Module

\subsubsection{Syntax}

\paragraph{Exported Constants:}
\begin{itemize}
    \item \texttt{DEFAULT\_THEME: char} -- Specifies the default theme for the game UI (e.g., light mode).
    \item \texttt{FONT\_STYLE: char} -- Default font style used across UI elements.
\end{itemize}

\paragraph{Exported Access Programs:}
\begin{itemize}
    \item \texttt{updateCardDisplay(playerId: char, cards: sequence of char) $\to$ void}
    \item \texttt{showTurnIndicator(playerId: char) $\to$ void}
    \item \texttt{displayMessage(message: char) $\to$ void}
\end{itemize}

\subsubsection{Semantics}

\paragraph{State Variables:}
\begin{itemize}
    \item \texttt{displayedCards: sequence of char} -- Tracks the cards currently visible for each player.
    \item \texttt{turnIndicator: char} -- Indicates which player's turn it is.
    \item \texttt{messageQueue: sequence of char} -- Stores pending notifications or chat messages to be displayed.
\end{itemize}

\paragraph{Environment Variables:}
\begin{itemize}
    \item \texttt{theme: char} -- Specifies the current visual theme (e.g., light/dark mode).
    \item \texttt{screenResolution: sequence of \mathbb{N}} -- The resolution of the device being used.
\end{itemize}

\paragraph{Assumptions:}
\begin{itemize}
    \item The UI module assumes that game state updates from the Game Logic Module are reliable.
    \item All required assets are preloaded by the Asset Management Module.
\end{itemize}

\paragraph{Access Routine Semantics:}
\begin{itemize}
    \item \texttt{updateCardDisplay(playerId: char, cards: sequence of char):}
    \begin{itemize}
        \item \textbf{Transition:} Updates the player’s visible hand to reflect the current state of their cards.
    \end{itemize}
    \item \texttt{showTurnIndicator(playerId: char):}
    \begin{itemize}
        \item \textbf{Transition:} Highlights the current player's turn using visual indicators.
    \end{itemize}
    \item \texttt{displayMessage(message: char):}
    \begin{itemize}
        \item \textbf{Transition:} Displays a notification or chat message on the game screen.
    \end{itemize}
\end{itemize}

\paragraph{Local Functions:}
\begin{itemize}
    \item \texttt{applyTheme(themeId: char) $\to$ void:}
    \begin{itemize}
        \item \textbf{Description:} Configures and applies the selected theme for the game UI.
    \end{itemize}
    \item \texttt{renderMessageQueue() $\to$ void:}
    \begin{itemize}
        \item \textbf{Description:} Processes and displays pending messages in the queue.
    \end{itemize}
    \item \texttt{adjustUILayout() $\to$ void:}
    \begin{itemize}
        \item \textbf{Description:} Dynamically adjusts the layout based on the screen resolution and device type.
    \end{itemize}
\end{itemize}

\newpage

\subsection{Save/Load Module}

\subsubsection{Module}
Save/Load (SL)

\subsubsection{Uses}
File Management Module, Game Logic Module, Multiplayer Module

\subsubsection{Syntax}

\paragraph{Exported Constants:}
\begin{itemize}
    \item \texttt{SAVE\_FORMAT: char} -- Specifies the file format used for saving (e.g., JSON).
\end{itemize}

\paragraph{Exported Access Programs:}
\begin{itemize}
    \item \texttt{saveGameState(gameId: char, state: char) $\to$ boolean}
    \item \texttt{loadGameState(gameId: char) $\to$ char}
\end{itemize}

\subsubsection{Semantics}

\paragraph{State Variables:}
\begin{itemize}
    \item \texttt{saveDirectory: char} -- Tracks the directory where game states are saved.
\end{itemize}

\paragraph{Environment Variables:}
\begin{itemize}
    \item \texttt{availableSpace: \mathbb{N}} -- Tracks available disk space for saving.
\end{itemize}

\paragraph{Assumptions:}
\begin{itemize}
    \item All files are accessible and writable by the system.
    \item The file management system is reliable.
\end{itemize}

\paragraph{Access Routine Semantics:}
\begin{itemize}
    \item \texttt{saveGameState(gameId: char, state: char):}
    \begin{itemize}
        \item \textbf{Transition:} Saves the current game state to the specified location.
    \end{itemize}
    \item \texttt{loadGameState(gameId: char):}
    \begin{itemize}
        \item \textbf{Transition:} Loads the game state from the specified file.
    \end{itemize}
\end{itemize}

\paragraph{Local Functions:}
\begin{itemize}
    \item \texttt{compressSaveData(data: char) $\to$ char:}
    \begin{itemize}
        \item \textbf{Description:} Compresses the save data to reduce file size.
    \end{itemize}
    \item \texttt{validateSaveFile(file: char) $\to$ boolean:}
    \begin{itemize}
        \item \textbf{Description:} Ensures the save file format is correct.
    \end{itemize}
\end{itemize}

\newpage

\subsection{Output Module}

\subsubsection{Module}
Output (O)

\subsubsection{Uses}
Game Logic Module, Animation Module

\subsubsection{Syntax}

\paragraph{Exported Constants:}
None

\paragraph{Exported Access Programs:}
\begin{itemize}
    \item \texttt{renderGraphics(graphicsData: char) $\to$ void}
    \item \texttt{playSound(soundId: char) $\to$ void}
\end{itemize}

\subsubsection{Semantics}

\paragraph{State Variables:}
\begin{itemize}
    \item \texttt{currentFrame: char} -- Tracks the current frame being rendered.
\end{itemize}

\paragraph{Environment Variables:}
\begin{itemize}
    \item \texttt{graphicsSettings: char} -- Stores graphics settings (e.g., resolution).
\end{itemize}

\paragraph{Assumptions:}
\begin{itemize}
    \item Rendering and sound systems are operational.
\end{itemize}

\paragraph{Access Routine Semantics:}
\begin{itemize}
    \item \texttt{renderGraphics(graphicsData: char):}
    \begin{itemize}
        \item \textbf{Transition:} Processes and renders graphics on the display.
    \end{itemize}
    \item \texttt{playSound(soundId: char):}
    \begin{itemize}
        \item \textbf{Transition:} Plays the sound associated with the given ID.
    \end{itemize}
\end{itemize}

\paragraph{Local Functions:}
\begin{itemize}
    \item \texttt{adjustBrightness(level: \mathbb{N}) $\to$ void:}
    \begin{itemize}
        \item \textbf{Description:} Adjusts the brightness of the rendered graphics.
    \end{itemize}
\end{itemize}

\newpage

\subsection{Animation Module}

\subsubsection{Module}
Animation (A)

\subsubsection{Uses}
Output Module, Game Logic Module

\subsubsection{Syntax}

\paragraph{Exported Constants:}
\begin{itemize}
    \item \texttt{FRAME\_RATE: \mathbb{N}} -- Specifies the animation frame rate.
\end{itemize}

\paragraph{Exported Access Programs:}
\begin{itemize}
    \item \texttt{startAnimation(animationId: char) $\to$ void}
    \item \texttt{stopAnimation(animationId: char) $\to$ void}
\end{itemize}

\subsubsection{Semantics}

\paragraph{State Variables:}
\begin{itemize}
    \item \texttt{runningAnimations: sequence of char} -- Tracks currently running animations.
\end{itemize}

\paragraph{Environment Variables:}
\begin{itemize}
    \item \texttt{maxConcurrentAnimations: \mathbb{N}} -- Maximum animations allowed at once.
\end{itemize}

\paragraph{Assumptions:}
\begin{itemize}
    \item The output module can handle all animation rendering.
\end{itemize}

\paragraph{Access Routine Semantics:}
\begin{itemize}
    \item \texttt{startAnimation(animationId: char):}
    \begin{itemize}
        \item \textbf{Transition:} Begins the specified animation.
    \end{itemize}
    \item \texttt{stopAnimation(animationId: char):}
    \begin{itemize}
        \item \textbf{Transition:} Stops the specified animation.
    \end{itemize}
\end{itemize}

\paragraph{Local Functions:}
\begin{itemize}
    \item \texttt{interpolateFrames(animationId: char) $\to$ sequence of char:}
    \begin{itemize}
        \item \textbf{Description:} Generates intermediate frames for smooth animation.
    \end{itemize}
\end{itemize}



\bibliographystyle {plainnat}
\bibliography {../../../refs/References}

\newpage

\section{Appendix} \label{Appendix}

\wss{Extra information if required}

\newpage{}

\section*{Appendix --- Reflection}

\wss{Not required for CAS 741 projects}

The information in this section will be used to evaluate the team members on the
graduate attribute of Problem Analysis and Design.



\begin{enumerate}
  \item What went well while writing this deliverable? 
  \item What pain points did you experience during this deliverable, and how
    did you resolve them?
  \item Which of your design decisions stemmed from speaking to your client(s)
  or a proxy (e.g. your peers, stakeholders, potential users)? For those that
  were not, why, and where did they come from?
  \item While creating the design doc, what parts of your other documents (e.g.
  requirements, hazard analysis, etc), it any, needed to be changed, and why?
  \item What are the limitations of your solution?  Put another way, given
  unlimited resources, what could you do to make the project better? (LO\_ProbSolutions)
  \item Give a brief overview of other design solutions you considered.  What
  are the benefits and tradeoffs of those other designs compared with the chosen
  design?  From all the potential options, why did you select the documented design?
  (LO\_Explores)
\end{enumerate}


\end{document}