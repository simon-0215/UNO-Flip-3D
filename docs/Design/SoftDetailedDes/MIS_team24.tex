\documentclass[12pt, titlepage]{article}

\usepackage{amsmath, mathtools}

\usepackage[round]{natbib}
\usepackage{amsfonts}
\usepackage{amssymb}
\usepackage{graphicx}
\usepackage{colortbl}
\usepackage{xr}
\usepackage{hyperref}
\usepackage{longtable}
\usepackage{xfrac}
\usepackage{tabularx}
\usepackage{float}
\usepackage{siunitx}
\usepackage{booktabs}
\usepackage{multirow}
\usepackage[section]{placeins}
\usepackage{caption}
\usepackage{fullpage}

\hypersetup{
bookmarks=true,     % show bookmarks bar?
colorlinks=true,       % false: boxed links; true: colored links
linkcolor=red,          % color of internal links (change box color with linkbordercolor)
citecolor=blue,      % color of links to bibliography
filecolor=magenta,  % color of file links
urlcolor=cyan          % color of external links
}

\usepackage{array}

\externaldocument{../../SRS/SRS}



\begin{document}

\title{Module Interface Specification for \progname{UnoFlip3D}}

\author{\authname Mingyang Xu \\ Kevin Ishak \\ Jianhao Wei \\ Zain-Alabedeen Garada \\ Zheng Beng Liang}

\date{\today}

\maketitle

\pagenumbering{roman}

\section{Revision History}

\begin{tabularx}{\textwidth}{p{4cm}p{3cm}X}
\toprule {\bf Date} & {\bf Developer} & {\bf Notes}\\
\midrule
January 12th, 2025 & Kevin Ishak & Initialize template, add rough draft of section 3,4 and 5\\
January 13th, 2025 & Jianhao Wei & Add modules into section 6\\
January 13th, 2025 & Zain-Alabedeen Garada & Add modules into section 6\\
\bottomrule
\end{tabularx}

~\newpage

\section{Symbols, Abbreviations and Acronyms}

See SRS Documentation at \wss{give url}

\wss{Also add any additional symbols, abbreviations or acronyms}

\newpage

\tableofcontents

\newpage

\pagenumbering{arabic}

\section{Introduction}

UNO Flip is a modern twist on the traditional UNO card game, incorporating an innovative double-sided card deck with ”light” and ”dark” sides. Players are challenged to adapt their strategies dynamically as the game flips between these two modes. Our goal for this project is to design and develop a digital version of UNO Flip that emulates the physical gameplay experience while adding features like automated rule enforcement, multiplayer support, and interactive animations.

Complementary documents include the System Requirement Specifications and Module Guide. The full documentation and implementation can be found at \url{https://github.com/simon-0215/UNO-Flip-3D/tree/main} .


\section{Notation}

\wss{You should describe your notation.  You can use what is below as
  a starting point.}

The structure of the MIS for modules comes from \citet{HoffmanAndStrooper1995},
with the addition that template modules have been adapted from
\cite{GhezziEtAl2003}.  The mathematical notation comes from Chapter 3 of
\citet{HoffmanAndStrooper1995}.  For instance, the symbol := is used for a
multiple assignment statement and conditional rules follow the form $(c_1
\Rightarrow r_1 | c_2 \Rightarrow r_2 | ... | c_n \Rightarrow r_n )$.

The following table summarizes the primitive data types used by \progname. 

\begin{center}
\renewcommand{\arraystretch}{1.2}
\noindent 
\begin{tabular}{l l p{7.5cm}} 
\toprule 
\textbf{Data Type} & \textbf{Notation} & \textbf{Description}\\ 
\midrule
character & char & a single symbol or digit\\
integer & $\mathbb{Z}$ & a number without a fractional component in (-$\infty$, $\infty$) \\
natural number & $\mathbb{N}$ & a number without a fractional component in [1, $\infty$) \\
real & $\mathbb{R}$ & any number in (-$\infty$, $\infty$)\\
\bottomrule
\end{tabular} 
\end{center}

\noindent
The specification of \progname \ uses some derived data types: sequences, strings, and
tuples. Sequences are lists filled with elements of the same data type. Strings
are sequences of characters. Tuples contain a list of values, potentially of
different types. In addition, \progname \ uses functions, which
are defined by the data types of their inputs and outputs. Local functions are
described by giving their type signature followed by their specification.

\section{Module Decomposition}

The following table is taken directly from the Module Guide document for this project.

\begin{table}[h!]
\centering
\begin{tabular}{p{0.3\textwidth} p{0.6\textwidth}}
\toprule
\textbf{Level 1} & \textbf{Level 2}\\
\midrule

\multirow{3}{0.3\textwidth}{Hardware-Hiding} & Device Input\\
& Audio Output\\
& Screen Rendering\\
\midrule

\multirow{4}{0.3\textwidth}{Behaviour-Hiding} & Game Rules Logic\\
& Turn Management\\
& Card Management\\
& Player Interaction\\
\midrule

\multirow{2}{0.3\textwidth}{Software Decision} 
& Networking Protocol\\
& Multiplayer System\\
\bottomrule

\end{tabular}
\caption{Module Hierarchy}
\label{TblMH}
\end{table}

\newpage
~\newpage


%Section 6

\section{MIS of Turn Management Module} 

\subsection{Module}
\hspace{1.5em}Turn Management

\subsection{Uses}
\hspace{1.5em}Multiplayer, UI, AssetManagement, Server

\subsection{Syntax}

\subsubsection{Exported Constants}
\hspace{1.5em}None

\subsubsection{Exported Access Programs}

\begin{itemize}
\item validateMove(playerId, cardid)
\item endTurn(playerId)
\item shuffleDeck()
\item drawCard(playerId)
\end{itemize}

\subsection{Semantics}

\subsubsection{State Variables}
\begin{itemize}
\item currentPlayer: Tracks the player whose turn it is
\item deck: Represents the stack of remaining cards in the game.
\item discardPile: Stores played cards
\item playerHands: Stores each player's cards
\end{itemize}

\subsubsection{Environment Variables}
\begin{itemize}
\item maxPlayers: Maximum number of players allowed in a game
\item flipEnabled: Boolean to toggle the flip functionality
\end{itemize}

\subsubsection{Assumptions}
\begin{itemize}
\item The number of players, game rules, player restrictions are preloaded
\item The game environment is known
\end{itemize}

\subsubsection{Access Routine Semantics}

\begin{itemize}
\item validateMove(playerId, card)\\
Transition: Checks if a move is valid

\item endTurn(playerId)\\
Transition: Ends the current player's turn and starts the next

\item shuffleDeck()\\
Transition: Randomizes the card deck

\item drawCard(playerId)\\
Transition: Adds a card to the specified player’s hand

\end{itemize}


\subsubsection{Local Functions}
\begin{itemize}
\item shuffleProcess()\\
\textbf{Description:} Contain the random algorithm to shuffle the deck
\item CardModifier()\\
\textbf{Description:} Contain algorithm to draw different card to screen
\end{itemize}


%Section 7

\section{MIS of Multiplayer Module} 

\subsection{Module}
\hspace{1.5em}Multiplayer

\subsection{Uses}
\hspace{1.5em}GameLogic, Server

\subsection{Syntax}

\subsubsection{Exported Constants}
\hspace{1.5em}None

\subsubsection{Exported Access Programs}
\begin{itemize}
\item createGameRoom(playerId, roomSettings)
\item joinGameRoom(playerId, roomId)
\item broadcastUpdate(gameId, update)
\end{itemize}

\subsection{Semantics}

\subsubsection{State Variables}
\begin{itemize}
\item activeGames: Tracks all ongoing game sessions.
\item connectedPlayers: List of currently connected players.
\end{itemize}


\subsubsection{Environment Variables}
\begin{itemize}
\item serverIP: IP address of the game server.
\item timeoutLimit: Time limit for a player to respond during their turn.
\end{itemize}

\subsubsection{Assumptions}
\begin{itemize}
\item The connection to other machines can be established successfully
\item The encryption and decryption methods are known
\end{itemize}

\subsubsection{Access Routine Semantics}

\begin{itemize}
\item createGameRoom(playerId, roomSettings)\\
Creates a new game room


\item joinGameRoom(playerId, roomId)\\
Adds a player to an existing room

\item broadcastUpdate(gameId, update)\\
Sends game state updates to all players in a room.

\end{itemize}


\subsubsection{Local Functions}
\begin{itemize}
\item encryption(information)\\
\textbf{Description:} Contain encryption algorithm to encrypt data before sending
\item decryption(information)\\ 
\textbf{Description:} Contain decryption algorithm to decrypt data after receiving
\end{itemize}



\bibliographystyle {plainnat}
\bibliography {../../../refs/References}

\newpage

\section{Appendix} \label{Appendix}

\wss{Extra information if required}

\newpage{}

\section*{Appendix --- Reflection}

\wss{Not required for CAS 741 projects}

The information in this section will be used to evaluate the team members on the
graduate attribute of Problem Analysis and Design.



\begin{enumerate}
  \item What went well while writing this deliverable? 
  \item What pain points did you experience during this deliverable, and how
    did you resolve them?
  \item Which of your design decisions stemmed from speaking to your client(s)
  or a proxy (e.g. your peers, stakeholders, potential users)? For those that
  were not, why, and where did they come from?
  \item While creating the design doc, what parts of your other documents (e.g.
  requirements, hazard analysis, etc), it any, needed to be changed, and why?
  \item What are the limitations of your solution?  Put another way, given
  unlimited resources, what could you do to make the project better? (LO\_ProbSolutions)
  \item Give a brief overview of other design solutions you considered.  What
  are the benefits and tradeoffs of those other designs compared with the chosen
  design?  From all the potential options, why did you select the documented design?
  (LO\_Explores)
\end{enumerate}


\end{document}