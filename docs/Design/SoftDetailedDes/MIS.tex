\documentclass[12pt, titlepage]{article}

\usepackage{amsmath, mathtools}
\usepackage[round]{natbib}
\usepackage{amsfonts}
\usepackage{amssymb}
\usepackage{graphicx}
\usepackage{colortbl}
\usepackage{xr}
\usepackage{hyperref}
\usepackage{longtable}
\usepackage{xfrac}
\usepackage{tabularx}
\usepackage{float}
\usepackage{siunitx}
\usepackage{booktabs}
\usepackage{multirow}
\usepackage[section]{placeins}
\usepackage{caption}
\usepackage{fullpage}
\usepackage{array}

\newcommand{\progname}[1]{\textit{#1}}


% Change tracking macros
\usepackage[normalem]{ulem} % allows \sout
\usepackage{xcolor}
\newcommand{\removed}[1]{\textcolor{red}{\sout{#1}}}
\newcommand{\added}[1]{\textcolor{green}{#1}}

\hypersetup{
    bookmarks=true,
    colorlinks=true,
    linkcolor=blue,
    citecolor=blue,
    filecolor=magenta,
    urlcolor=cyan
}

\externaldocument{../../SRS/SRS}

\begin{document}

\title{Module Interface Specification for \progname{Uno Flip Remix}}


\author{\authname Team 24 \\ \\ Mingyang Xu \\ Kevin Ishak \\ Jianhao Wei \\ Zain-Alabedeen Garada \\ \removed{Zheng Beng Liang}}

\date{April 2, 2025}

\maketitle

\pagenumbering{roman}

\section{Revision History}

\begin{tabularx}{\textwidth}{p{4cm}p{3cm}X}
\toprule {\bf Date} & {\bf Developer} & {\bf Notes}\\
\midrule
January 12th, 2025 & Kevin Ishak & Initialize template, add rough draft of section 3,4 and 5\\
January 13th, 2025 & Jianhao Wei & Add modules into section 6\\
January 13th, 2025 & Zain-Alabedeen Garada & Add modules into section 6\\
January 17th, 2025 & Zheng Bang Liang & Added all modules for Behavior Hiding Modules\\
January 17th, 2025 & Jianhao Wei, Kevin Ishak, Zain-Alabedeen Garada & Modify behavior hiding modules that Zheng added and add the rest of the MIS modules \\
January 17th, 2025 & Jianhao Wei & Modify section 3 and wrote section 4, 5, 16. Communicate with other members and wrote section 17\\
January 28th, 2025 & Jianhao Wei & Make changes based on peer feedbacks. Please see commits and issue trackers for detail\\
\added{April 1st, 2025} & \added{Kevin Ishak} & \added{Edits based on TA feedback for the final report revision}\\
\bottomrule
\end{tabularx}

\newpage

\section{Symbols, Abbreviations and Acronyms}

\added{The following table summarizes the symbols and abbreviations used throughout this document. These definitions are consistent with those listed in the Module Guide (MG). For the complete reference, see the full MG at \href{https://github.com/simon-0215/UNO-Flip-3D/blob/main/docs/Design/SoftArchitecture/MG.pdf}{this link}.}

\renewcommand{\arraystretch}{1.2}
\begin{longtable}{p{0.25\textwidth} p{0.65\textwidth}}
\toprule
\textbf{Symbol / Abbreviation} & \textbf{Definition} \\
\midrule
AC & Anticipated Change \\
UC & Unlikely Change \\
MG & Module Guide \\
MIS & Module Interface Specification \\
SRS & Software Requirements Specification \\
UI & User Interface \\
UDP & User Datagram Protocol \\
TCP & Transmission Control Protocol \\
DAG & Directed Acyclic Graph \\
HUD & Heads-Up Display \\
UNO Flip & A variant of the classic UNO game featuring a two-sided deck \\
\bottomrule
\end{longtable}

\newpage

\tableofcontents

\newpage

\pagenumbering{arabic}

\section{Introduction}

UNO Flip is a modern twist on the traditional UNO card game, incorporating an innovative double-sided card deck with “light” and “dark” sides. Players must adapt their strategy dynamically as the game flips between these two sides, creating unpredictable and engaging game play.

The goal of this project is to develop a digital version of UNO Flip that faithfully replicates the physical game play experience while introducing improvements such as rule automation, multiplayer networking, real-time animations, and a modern user interface.

This Module Interface Specification (MIS) provides detailed interface specifications for each software module described in the Module Guide (MG). It serves as a design reference for implementation and testing.

\added{Complementary documents include:}
\begin{itemize}
    \item \added{\href{https://github.com/simon-0215/UNO-Flip-3D/tree/main/docs/SRS-Volere}{The Software Requirements Specification (SRS)}, which defines functional and non-functional system requirements.}
    \item \added{\href{https://github.com/simon-0215/UNO-Flip-3D/blob/main/docs/Design/SoftArchitecture/MG.pdf}{The Module Guide (MG)}, which outlines the modular decomposition and design philosophy.}
\end{itemize}

\section{Notation}

The structure of the MIS for modules follows the documentation standards described in Hoffman and Strooper \cite{HoffmanAndStrooper1995}, with adaptations based on Ghezzi et al. \cite{GhezziEtAl2003}. The mathematical notation follows conventions from Chapter 3 of Hoffman and Strooper (1995). This MIS also builds on the SFWRENG 4G06 GitHub template, available at \href{https://github.com/smiths/capTemplate/tree/main/docs/Design/SoftDetailedDes}{this link}.


\added{The following tables summarize the primitive, object, and domain-specific data types used by the UNO Flip 3D software. These types form the basis for module specifications throughout this document.}

\vspace{0.5em}
\noindent
\textbf{Primitive Data Types}

\begin{center}
\renewcommand{\arraystretch}{1.2}
\begin{tabular}{l l p{8cm}} 
\toprule 
\textbf{Data Type} & \textbf{Notation} & \textbf{Description}\\ 
\midrule
Boolean & boolean & Logical value representing \texttt{true} or \texttt{false}. \\
Integer & int & Whole numbers in the range $[-2^{63}, 2^{63}-1]$.\\
Floating Point & float & Real numbers with fractional components using IEEE 754 32-bit representation.\\
Serialized Data Stream & serializedData & Binary-encoded data used for inter-module or inter-device communication.\\
\bottomrule
\end{tabular} 
\end{center}

\vspace{1em}
\noindent 
\newpage
\textbf{Derived Object Data Types}

\begin{center}
\renewcommand{\arraystretch}{1.2}
\begin{tabular}{l l p{8cm}} 
\toprule 
\textbf{Object Type} & \textbf{Notation} & \textbf{Description}\\ 
\midrule
String & String & A sequence of Unicode characters.\\
Generic Array & Array[Type] & A collection of elements of the specified type.\\
Dictionary & dictionary & Key-value mapping where each key corresponds to a specific element.\\
Graphics Object & GraphicObject & Object describing visual representation, used by the UI module.\\
\bottomrule
\end{tabular} 
\end{center}

\vspace{1em}
\noindent
\textbf{\added{Domain-Specific Data Types}}

\begin{center}
\renewcommand{\arraystretch}{1.2}
\begin{tabular}{l l p{8cm}} 
\toprule 
\textbf{Custom Type} & \textbf{Notation} & \textbf{Description}\\ 
\midrule
Card & Card & Represents an individual card with color, number, and action attributes.\\
Deck & Deck & A stack of \texttt{Card} objects supporting draw and shuffle operations.\\
Player & Player & Entity with a hand of cards, name, and status.\\
Game State & GameState & Tracks all gameplay data: turn order, decks, hands, discard piles, etc.\\
Move & Move & Encodes a player's chosen card and action.\\
Turn Direction & TurnDirection & Enum representing clockwise or counterclockwise direction.\\
Action Type & ActionType & Enum for card actions: \texttt{Draw2}, \texttt{Skip}, \texttt{Flip}, etc.\\
Color & Color & Enum for card colors: \texttt{Red}, \texttt{Blue}, \texttt{Green}, \texttt{Yellow}, \texttt{Wild}.\\
\bottomrule
\end{tabular} 
\end{center}

\vspace{0.5em}
\noindent
UNO Flip 3D uses functions, which are defined by the data types of their inputs and outputs. Local functions are described by giving their type signature followed by their specification.

\section{Module Decomposition}

\added{This section outlines the modular decomposition of the UNO Flip Remix system, categorized into three types: Hardware-Hiding, Behaviour-Hiding, and Software Decision modules. These categories follow the modular design principles established in the Module Guide (MG). Each module listed below is a leaf module from the MG hierarchy and is associated with an interface specification in this document.}

\subsection{Hardware-Hiding Modules}

\begin{itemize}
\item \hyperref[BSM]{\textbf{Backend/Server Module:}} Serves as a virtual hardware abstraction layer between Unity clients and the network server. It is responsible for input/output processing and enabling communication across clients in a multiplayer environment.
\end{itemize}

\subsection{Behaviour-Hiding Modules}

\begin{itemize}
\item \hyperref[CEM]{\textbf{Card Effect Module:}} Executes the effects of special cards and updates the game state accordingly.
\item \hyperref[TMM]{\textbf{Turn Management Module:}} Manages the order of player turns, including handling special conditions like “Reverse” or “Skip” cards.
\item \hyperref[UIM]{\textbf{User Interface Module:}} Displays the game state to the user and accepts user inputs through interactive UI elements.
\item \hyperref[SLM]{\textbf{Save/Load Module:}} Allows saving and restoring the game state from persistent storage.
\item \hyperref[AM]{\textbf{Animation Module:}} Provides animations for card movements, flips, and visual effects that enhance gameplay.
\item \hyperref[OM]{\textbf{Output Module:}} Renders visual and textual game outputs, including scoreboards and notifications.
\end{itemize}

\subsection{Software Decision Modules}

\begin{itemize}
\item \hyperref[MNM]{\textbf{Multiplayer Networking Module:}} Handles matchmaking, game state synchronization, and reliable communication between clients using Unity and a TCP-based server.
\item \hyperref[VOM]{\textbf{Verification Output Module:}} Validates game logic and output correctness, ensuring rule compliance before and after each turn.
\item \hyperref[IM]{\textbf{Input Module:}} Converts raw user inputs into standardized data structures usable by the game logic and UI.
\item \hyperref[CLM]{\textbf{Control Logic Module:}} Manages the overall game control flow, including transitions between states, coordination between modules, and input/output sequencing.
\end{itemize}

\newpage

\section{MIS of Backend Server Module}
\label{BSM}

\subsection{Module}
\hspace{1.5em}BackendServer

\subsection{Uses}
\hspace{1.5em}InputModule, OutputModule, MultiplayerNetworkingModule

\subsection{Syntax}

\subsubsection{Exported Constants}
\begin{itemize}
    \item MAX\_CLIENTS: \texttt{int} — Maximum number of clients supported by the server.
    \item DEFAULT\_PORT: \texttt{int} — Default port used for incoming socket connections.
\end{itemize}

\subsubsection{Exported Access Programs}
\begin{itemize}
    \item startServer(port: int) $\rightarrow$ boolean
    \item acceptConnection() $\rightarrow$ Connection
    \item broadcastMessage(msg: String) $\rightarrow$ void
    \item shutdownServer() $\rightarrow$ void
\end{itemize}

\subsection{Semantics}

\subsubsection{State Variables}
\begin{itemize}
    \item serverSocket: Stores the active server socket instance.
    \item clients: A list of active client connections.
    \item gameSessions: Stores game session state mapped to connected players.
\end{itemize}

\subsubsection{Environment Variables}
\begin{itemize}
    \item OSNetworkStack: The operating system’s network protocol stack (e.g., TCP/IP).
    \item UnityEngine.Platform: Unity’s internal networking environment used to interface with the server.
\end{itemize}

\subsubsection{Assumptions}
\begin{itemize}
    \item The host machine allows socket binding on the specified port.
    \item Unity clients conform to the expected messaging protocol.
\end{itemize}

\subsubsection{Access Routine Semantics}

\begin{itemize}
    \item startServer(port) $\rightarrow$ boolean\\
    \textbf{Transition:} Binds a socket to the specified port and begins listening for incoming client connections.\\
    \textbf{Output:} Returns \texttt{true} if the server starts successfully; otherwise, \texttt{false}.

    \item acceptConnection() $\rightarrow$ Connection\\
    \textbf{Transition:} Accepts a new client connection request and appends it to the active client list.\\
    \textbf{Output:} Returns a new \texttt{Connection} object representing the connected client.

    \item broadcastMessage(msg) $\rightarrow$ void\\
    \textbf{Transition:} Sends the message \texttt{msg} to all connected clients via TCP.

    \item shutdownServer() $\rightarrow$ void\\
    \textbf{Transition:} Closes all active client connections and releases the server socket.
\end{itemize}

\subsubsection{Local Functions}

\begin{itemize}
    \item validateMessageFormat(msg: String) $\rightarrow$ boolean\\
    \textbf{Description:} Checks if the incoming message string conforms to the expected JSON protocol.

    \item removeInactiveClients() $\rightarrow$ void\\
    \textbf{Description:} Iterates over the client list and removes disconnected or timed-out clients.

    \item logConnectionEvent(event: String) $\rightarrow$ void\\
    \textbf{Description:} Appends server-side events to a log for debugging and traceability.
\end{itemize}

\section{MIS of Card Effect Module} 
\label{CEM}

\subsection{Module}
\hspace{1.5em}Card Effect

\subsection{Uses}
\hspace{1.5em}Hardwire Hiding

\subsection{Syntax}

\subsubsection{Exported Constants}
\begin{itemize}
\item DRAW\_TWO\_EFFECT: Specifies the effect identifier for a "Draw Two" card.
\item SKIP\_TURN\_EFFECT: Specifies the effect identifier for a "Skip" card.
\item FLIP\_DECK\_EFFECT: Specifies the effect identifier for a "Flip" card.
\end{itemize}

\subsubsection{Exported Access Programs}

\begin{itemize}
\item reverseDirection(playerId: int)
\item skipTurn(playerId: int)
\item triggerDrawCards(playerId: int, cardCount: Int)
\item flipDeck()
\end{itemize}

\subsection{Semantics}

\subsubsection{State Variables}
\begin{itemize}
\item currentEffect: Track the current effect being applied
\item effectQueue: Store the effects that are waiting to be applied
\end{itemize}

\subsubsection{Environment Variables}
\begin{itemize}
    \item The special card type that current game environment allowed.
\end{itemize}

\subsubsection{Assumptions}
\begin{itemize}
\item All card effects are predefined.
\item The "Flip" card effect toggles the entire game state between "light" and "dark" sides.
\end{itemize}

\subsubsection{Access Routine Semantics}

\begin{itemize}
\item reverseDirection(playerId: int, previousPlayerId: int): $\rightarrow$ void\\
\textbf{Transition:} Reverse the direction the game is being played to the previous player by re-assigning the previous player next opportunity

\item skipTurn(playerId: int, nextPlayerId: int) $\rightarrow$ void\\
\textbf{Transition:} Skip the opportunity for the specified player to play and assignment the opportunity to the next player

\item triggerDrawCards(playerId: int, cardCount: Int) $\rightarrow$ void\\
\textbf{Transition:} Let the player specified to draw another card into their database and add the card count to their totalPower variable.

\item flipDeck() $\rightarrow$ void\\
\textbf{Transition:} Changes the deckSide variable and updates the game state to reflect the flipped deck.

\end{itemize}


\subsubsection{Local Functions}
\begin{itemize}
\item calculateNextPlayer(direction: String) $\rightarrow$ int\\
\textbf{Description:} Determines the next player in the turn sequence after applying a "Skip" or "Reverse" effect.

\item applyChainEffect(effectQueue: Array[String]) $\rightarrow$ void\\
\textbf{Description:} Resolves multiple card effects in sequence defined in the input string array

\item toggleDeckSide() $\rightarrow$ void\\
\textbf{Description:} Switches the game state between "light" and "dark" sides during a "Flip" card effect.
\end{itemize}



\section{MIS of Turn Management Module} 
\label{TMM}

\subsection{Module}
\hspace{1.5em}Turn Management

\subsection{Uses}
\hspace{1.5em}Input, Card Effect

\subsection{Syntax}

\subsubsection{Exported Constants}
\hspace{1.5em}None

\subsubsection{Exported Access Programs}

\begin{itemize}
\item validateMove(playerId: int, cardid: int)
\item endTurn(playerId: int)
\item shuffleDeck()
\item drawCard(playerId: int)
\end{itemize}

\subsection{Semantics}

\subsubsection{State Variables}
\begin{itemize}
\item currentPlayer: Tracks the player whose turn it is
\item deck: Represents the stack of remaining cards in the game.
\item discardPile: Stores played cards
\item playerHands: Stores each player's cards
\end{itemize}

\subsubsection{Environment Variables}
\begin{itemize}
\item maxPlayers: Maximum number of players allowed in a game
\item flipEnabled: Boolean to toggle the flip functionality
\end{itemize}

\subsubsection{Assumptions}
\begin{itemize}
\item The number of players, game rules, player restrictions are preloaded
\item The game environment is known
\end{itemize}

\subsubsection{Access Routine Semantics}

\begin{itemize}
\item validateMove(playerId: int, cardId: int) $\rightarrow$ boolean\\
\textbf{Output:} Checks if a move is valid

\item endTurn(playerId:int) $\rightarrow$ void\\
\textbf{Transition:} Ends the current player's turn and starts the next

\item shuffleDeck() $\rightarrow$ void\\
\textbf{Transition:} Randomizes the card deck

\item drawCard(playerId: int) $\rightarrow$ void\\
\textbf{Transition:} Adds a card to the specified player’s hand

\end{itemize}


\subsubsection{Local Functions}
\begin{itemize}
\item shuffleProcess(original: Array[String]) $\rightarrow$ Array[String]\\
\textbf{Description:} Contain the random algorithm to shuffle the deck
\item CardModifier(cardId: int) $\rightarrow$ void\\
\textbf{Description:} Contain algorithm to draw different card to screen
\end{itemize}


\section{MIS of User Interface Module} 
\label{UIM}

\subsection{Module}
\hspace{1.5em}User Interface

\subsection{Uses}
\hspace{1.5em}Output, Turn Management

\subsection{Syntax}

\subsubsection{Exported Constants}
\begin{itemize}
\item DEFAULT\_THEME: Specifies the default theme for the game UI (e.g., light mode).
\item FONT\_STYLE: Default font style used across UI elements.
\item ASSET\_PATH: Directory path where assets are stored
\item DEFAULT\_CARD\_SPRITE: Specifies the default card sprite to use if none is provided.

\end{itemize}

\subsubsection{Exported Access Programs}

\begin{itemize}
\item updateCardDisplay(playerId: int, cardId: int)
\item showTurnIndicator(playerId: int)
\item displayMessage(message: String)
\item loadScene(type: String)
\item loadAsset(assetName: String)
\item unloadAsset(assetName: String)
\item playSound(effectName: String)
\end{itemize}

\subsection{Semantics}

\subsubsection{State Variables}
\begin{itemize}
\item displayedCards: Tracks the cards currently visible for each player.
\item turnIndicator: Indicates which player’s turn it is.
\item messageQueue: Stores pending notifications or chat messages to be displayed.
\item theme: Specifies the current visual theme in light mode or dark mode.
\item loadedAssets: Tracks assets currently loaded into memory.
\item audioSettings: Stores configuration for playing audio
\item assetCache: Cache for frequently accessed assets to improve performance.
\item assetDirectory: Path to the directory containing all assets 

\end{itemize}

\subsubsection{Environment Variables}
\begin{itemize}
\item The resolution of the device being used.
\end{itemize}

\subsubsection{Assumptions}
\begin{itemize}
\item The UI module assumes that game state updates from the multiplayer networking and turn management modules are reliable.
\item All required assets are preloaded by the Save/Load module.
\item Multiplayer synchronization ensures accurate real-time updates across all connected devices.
\item All assets are correctly named and stored in the specified directory.
\item The module assumes sufficient memory and storage are available for caching assets.
\item Dependencies for visual and audio formats are preinstalled on the system.
\end{itemize}

\subsubsection{Access Routine Semantics}

\begin{itemize}
\item updateCardDisplay(playerId: int, cardId: int) $\rightarrow$ void\\
\textbf{Transition:} Updates the player’s visible hand to reflect the current state of their cards.

\item showTurnIndicator(playerId: int) $\rightarrow$ void\\
\textbf{Transition:} Highlights the current player's turn using visual indicators.

\item displayMessage(message: String) $\rightarrow$ void\\
\textbf{Transition:} Displays a notification or chat message on the game screen.

\item loadScene(type: String) $\rightarrow$ void\\
\textbf{Transition:} Load specific type of background with animation to the user interface

\item loadAsset(assetName: String) $\rightarrow$ void\\
\textbf{Transition:} Loads the specified asset from the asset directory into memory and returns a reference.

\item unloadAsset(assetName: String) $\rightarrow$ void\\
\textbf{Transition:} Removes the specified asset from memory to free up resources.

\item playSound(effectName: String) $\rightarrow$ void\\
\textbf{Transition:} Plays the specified sound effect from the audio assets directory.

\end{itemize}


\subsubsection{Local Functions}
\begin{itemize}
\item applyTheme(themeId: int) $\rightarrow$ void\\
\textbf{Description:} Configures and applies the selected theme for the game UI.

\item renderMessageQueue(messages: Array[String]) $\rightarrow$ void\\
\textbf{Description:} Processes and displays pending messages in the queue.

\item adjustUILayout() $\rightarrow$ void\\
\textbf{Description:} Dynamically adjusts the layout based on the screen resolution and device type.

\item cacheAsset(assetName: String) $\rightarrow$ void\\
\textbf{Description:} Adds the specified asset to the cache for quick retrieval.

\item clearCache() $\rightarrow$ void\\
\textbf{Description:} Clears the asset cache to free up memory

\item validateAsset(assetName: String) $\rightarrow$ void\\
\textbf{Description:} Checks if the specified asset exists and is accessible.

\end{itemize}





\section{MIS of Save/Load Module} 
\label{SLM}

\subsection{Module}
\hspace{1.5em}Save/Load

\subsection{Uses}
\hspace{1.5em}Hardwire Hiding

\subsection{Syntax}

\subsubsection{Exported Constants}
\hspace{1.5em}None

\subsubsection{Exported Access Programs}

\begin{itemize}
\item save(info: String, description: String)
\item retrieve(description: String)
\item delete(description: String)
\item changeDesc(originalDesc: String, updateDesc: String)
\end{itemize}

\subsection{Semantics}

\subsubsection{State Variables}
\begin{itemize}
\item ifFull: Track if the database is full
\item dict: The dictionary that stores the array index correspond with descriptions
\item infoArray: The array that stores all the information
\end{itemize}

\subsubsection{Environment Variables}
\hspace{1.5em}None

\subsubsection{Assumptions}
\hspace{1.5em}The string and description stored does not contain any special characters

\subsubsection{Access Routine Semantics}

\begin{itemize}
\item save(info: String, description: String) $\rightarrow$ void\\
\textbf{Transition:} Save the information into the database with description

\item retrieve(description: String) $\rightarrow$ String\\
\textbf{Output:} Return the information by its description

\item delete(description: String) $\rightarrow$ void\\
\textbf{Transition:} Delete the information in the database by its description

\item changeDesc(originalDesc: String, updateDesc: String) $\rightarrow$ void\\
\textbf{Transition:} change the description of a piece of information into another

\end{itemize}


\subsubsection{Local Functions}
\begin{itemize}
\item returnIndex(description: String) $\rightarrow$ int\\
\textbf{Description:} Return the index of the infoArray based on the description. 
\end{itemize}





\section{MIS of Animation Module} 
\label{AM}

\subsection{Module}
\hspace{1.5em}Animation

\subsection{Uses}
\hspace{1.5em}User Interface, Card Effect, Save/Load

\subsection{Syntax}

\subsubsection{Exported Constants}
\hspace{1.5em}None

\subsubsection{Exported Access Programs}

\begin{itemize}
\item move(cardId: int, distance: int, direction: String)
\item flip(cardId: int)
\item select(cardId: int)
\item appear(cardId: int)
\item disappear(cardId: int)
\end{itemize}

\subsection{Semantics}

\subsubsection{State Variables}
\begin{itemize}
\item cardSide: Track side the card is on
\item cardColor: Track the color of the card
\item cardPosition: Track the position of the card 
\item show: Track if the card is shown on the screen
\end{itemize}

\subsubsection{Environment Variables}
\hspace{1.5em}None

\subsubsection{Assumptions}
\hspace{1.5em}Each card has a unique id


\subsubsection{Access Routine Semantics}

\begin{itemize}
\item move(cardId: int, distance: int, direction: String) $\rightarrow$ void\\
\textbf{Transition:} Move the card with specific id by a set amount of pixels with horizontal or vertical direction

\item flip(cardId: int) $\rightarrow$ void\\
\textbf{Transition:} Flip the card with specific id to show the opposite face

\item select(cardId: int) $\rightarrow$ void\\
\textbf{Transition:} Show the animation when the card is selected by the user

\item appear(cardId: int) $\rightarrow$ void\\
\textbf{Transition:} Show the card with specific id to the user screen

\item disappear(cardId: int) $\rightarrow$ void\\
\textbf{Transition:} Make the card with specific id to disappear from the user screen
\end{itemize}


\subsubsection{Local Functions}
\begin{itemize}
\item getCardInfo(id: int) $\rightarrow$ void\\
\textbf{Description:} Get the info of the card to local state variables
\item applyVisualElements(id: int) $\rightarrow$ void\\
\textbf{Description:} Apply the visual effect to the user screen based on the id provided and update local state variables
\end{itemize}



\section{MIS of Output Module} 
\label{OM}

\subsection{Module}
\hspace{1.5em}Output

\subsection{Uses}
\hspace{1.5em}Card Effect

\subsection{Syntax}

\subsubsection{Exported Constants}
\hspace{1.5em}None

\subsubsection{Exported Access Programs}

\begin{itemize}
\item render(info: String, font: int, color: String, location: int)
\item showCardEffect(id: int, effectNum: int)
\end{itemize}

\subsection{Semantics}

\subsubsection{State Variables}
\hspace{1.5em}None

\subsubsection{Environment Variables}
\hspace{1.5em}None

\subsubsection{Assumptions}
\hspace{1.5em}Each card has a unique id


\subsubsection{Access Routine Semantics}

\begin{itemize}
\item render(info: String, font: int, color: String, location: int)
$\rightarrow$ void\\
\textbf{Transition:} Display the information onto the screen with the font, color and location specified

\item showCardEffect(id: int, effectNum: int) $\rightarrow$ void\\
\textbf{Transition:} Using Card Effect module to show flip, skip or draw two on specific card 	

\end{itemize}


\subsubsection{Local Functions}
\begin{itemize}
\item checkEdge(font: int, location: int) $\rightarrow$ boolean\\
\textbf{Description:} Check if the information displayed exceeds the boundary of the screen
\end{itemize}


\section{MIS of Multiplayer Networking Module} 
\label{MNM}

\subsection{Module}
\hspace{1.5em}Multiplayer Networking

\subsection{Uses}
\hspace{1.5em}Verification Output, Save/Load, Animation

\subsection{Syntax}

\subsubsection{Exported Constants}
\hspace{1.5em}serverID: The serial number of the game room upon user request
\subsubsection{Exported Access Programs}
\begin{itemize}
\item createGameRoom(playerId: int, roomSettings: Array[String])
\item joinGameRoom(playerId: int, roomId: int)
\item broadcastUpdate(gameId: int, update: String)
\end{itemize}

\subsection{Semantics}

\subsubsection{State Variables}
\begin{itemize}
\item activeGames: Tracks all ongoing game sessions.
\item connectedPlayers: List of currently connected players.
\end{itemize}


\subsubsection{Environment Variables}
\begin{itemize}
\item serverIP: IP address of the game server.
\item timeoutLimit: Time limit for a player to respond during their turn.
\end{itemize}

\subsubsection{Assumptions}
\begin{itemize}
\item The connection between server and other machines can be established successfully
\item The encryption and decryption methods are known
\end{itemize}

\subsubsection{Access Routine Semantics}

\begin{itemize}
\item createGameRoom(playerId: int, roomSettings: Array[String]) $\rightarrow$ void\\
\textbf{Transition:} Creates a new game room by a specific user with specific setting


\item joinGameRoom(playerId: int, roomId: int, publicKey: int) $\rightarrow$ int\\
\textbf{Transition:} Adds a specific player to an existing room by its ID and public key for encryption and decryption purpose purposes \\
\textbf{Output:} Return the public key of the server for encryption and decryption purposes

\item broadcastUpdate(gameId: int, update: String) $\rightarrow$ void\\
\textbf{Transition:} Sends game state updates to all players in a room.

\end{itemize}


\subsubsection{Local Functions}
\begin{itemize}
\item encryption(information: String, publicKey: int) $\rightarrow$ String\\
\textbf{Description:} Contain encryption algorithm to encrypt data before sending using public key from user
\item decryption(information: String, privateKey: int) $\rightarrow$ String\\ 
\textbf{Description:} Contain decryption algorithm to decrypt data after receiving using the private key of game room
\end{itemize}


\section{MIS of Verification Output Module}
\label{VOM}

\subsection{Module}
\hspace{1.5em}Verification Output

\subsection{Uses}
\hspace{1.5em}None

\subsection{Syntax}

\subsubsection{Exported Constants}
\hspace{1.5em}None

\subsubsection{Exported Access Programs}

\begin{itemize}
\item captureOutput(playerId: int, info: String)
\item validateOutput(info: String)
\end{itemize}

\subsection{Semantics}

\subsubsection{State Variables}
\begin{itemize}
\item outputBuffer: Temporarily store the incoming input received for later use
\item validatedOutput: Store the input that has been validated by the module for later transmission
\end{itemize}

\subsubsection{Environment Variables}
\begin{itemize}
\item the validation algorithm the device is running on
\end{itemize}

\subsubsection{Assumptions}
\begin{itemize}
\item All output devices conform to Unity's input standard.
\item The validation algorithm must make sure that there is no error or discrepancy occurring after the validation
\end{itemize}

\subsubsection{Access Routine Semantics}

\begin{itemize}
\item captureInput(info: String) $\rightarrow$ void\\
\textbf{Transition:} Capture and save the information into the output buffer

\item validateInput(info: String) $\rightarrow$ String, boolean\\
\textbf{Transition:} validate the output from the outputBuffer using existing algorithms\\
\textbf{Output:} Return the original output and a boolean indicating if the input can be validated


\end{itemize}


\subsubsection{Local Functions}
\begin{itemize}
\item algorithmDatabase(input: String, type: int) $\rightarrow$ String\\
\textbf{Description:} Contain the algorithm that converts the input string to the format that can be used by other modules and return the converted input string

\item serialization(input: String) $\rightarrow$ serializedData\\
\textbf{Description:} Contain the algorithm to convert the input string into serialized data for inter-module or internet communications
\end{itemize}






\section{MIS of Input Module} 
\label{IM}

\subsection{Module}
\hspace{1.5em}Input

\subsection{Uses}
\hspace{1.5em}Hardwire Hiding

\subsection{Syntax}

\subsubsection{Exported Constants}
\hspace{1.5em}None

\subsubsection{Exported Access Programs}

\begin{itemize}
\item captureInput(playerId: int, info: String)
\item validateInput(info: String)
\item convertInput(info: String, type: String)
\end{itemize}

\subsection{Semantics}

\subsubsection{State Variables}
\begin{itemize}
\item inputBuffer: Temporarily store the incoming input received for later use
\item validatedInput: Store the input that has been validated by the module for later transmission
\end{itemize}

\subsubsection{Environment Variables}
\begin{itemize}
\item The version of supporting device that the software is running on
\item the validation algorithm the device is running on
\end{itemize}

\subsubsection{Assumptions}
\begin{itemize}
\item All input devices conform to Unity's input standard.
\item The validation algorithm must make sure that there is no error or discrepancy occurring after the validation
\end{itemize}

\subsubsection{Access Routine Semantics}

\begin{itemize}
\item captureInput(playerId: int, info: String) $\rightarrow$ void\\
\textbf{Transition:} Capture and save the information into the input buffer

\item validateInput(info: String) $\rightarrow$ String, boolean\\
\textbf{Transition:} validate the input from the inputBuffer using existing algorithms\\
\textbf{Output:} Return the original input and a boolean indicating if the input can be validated

\item convertInput(action: String, type: String) $\rightarrow$ String\\
\textbf{Output:} Convert the input from validatedInput into specific format that can be used by other modules

\end{itemize}


\subsubsection{Local Functions}
\begin{itemize}
\item algorithmDatabase(input: String, type: int) $\rightarrow$ String\\
\textbf{Description:} Contain the algorithm that converts the input string to the format that can be used by other modules and return the converted input string

\item serialization(input: String) $\rightarrow$ serializedData\\
\textbf{Description:} Contain the algorithm to convert the input string into serialized data for inter-module or internet communications
\end{itemize}


\section{Exception Handling Strategies}
The exception handling is critical for our software since it directly impacts the user experience of our software. It is our responsibility to ensure that our customers have a good experience with our software. To prevent exception from happening in our software, we implement the following 4 strategies:
\begin{itemize}
    \item \textbf{Limit Erroneous User Input:} We design the user interface such that the user input is bounded within a certain range to limit erroneous user input that might crash the software. We also include the input verification in Input module to ensure all the information that passed to the software are legitimate
    \item \textbf{Wrap External Resources:} We have design all of our function in our modules to wrap the resources and libraries they use  from the global space of the software. This ensures that the exception in third-party software does not impact the integrity of own software.
    \item \textbf{Cleaning up resources:} We have implemented the mechanism to clean up unused resource promptly and reliably to make sure the exceptions do not occurs due to cache overload
    \item \textbf{Limiting Errors Instead of Handling Errors:} Instead of designing exception handling mechanism, we make sure our software is carefully designed and tested to reduce the chance of exception happening.
\end{itemize}
By implementing these strategies, we can reduce the chance for exception happening and limit the need of the exception handling mechanisms.


\newpage

\begin{thebibliography}{9}

\bibitem{HoffmanAndStrooper1995}
D. M. Hoffman and P. Strooper,
\textit{Software Design: With C++ and Java},
Addison-Wesley, 1995.

\bibitem{GhezziEtAl2003}
C. Ghezzi, M. Jazayeri, and D. Mandrioli,
\textit{Fundamentals of Software Engineering}, 2nd ed.,
Prentice Hall, 2003.

\end{thebibliography}

\end{document}