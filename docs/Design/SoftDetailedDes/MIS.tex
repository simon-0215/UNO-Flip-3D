\documentclass[12pt, titlepage]{article}

\usepackage{amsmath, mathtools}

\usepackage[round]{natbib}
\usepackage{amsfonts}
\usepackage{amssymb}
\usepackage{graphicx}
\usepackage{colortbl}
\usepackage{xr}
\usepackage{hyperref}
\usepackage{longtable}
\usepackage{xfrac}
\usepackage{tabularx}
\usepackage{float}
\usepackage{siunitx}
\usepackage{booktabs}
\usepackage{multirow}
\usepackage[section]{placeins}
\usepackage{caption}
\usepackage{fullpage}

\hypersetup{
bookmarks=true,     % show bookmarks bar?
colorlinks=true,       % false: boxed links; true: colored links
linkcolor=blue,          % color of internal links (change box color with linkbordercolor)
citecolor=blue,      % color of links to bibliography
filecolor=magenta,  % color of file links
urlcolor=cyan          % color of external links
}

\usepackage{array}

\externaldocument{../../SRS/SRS}



\begin{document}

\title{Module Interface Specification for \progname{UnoFlip3D}}

\author{\authname Team 24 \\ \\ Mingyang Xu \\ Kevin Ishak \\ Jianhao Wei \\ Zain-Alabedeen Garada \\ Zheng Beng Liang}

\date{January 17, 2025}

\maketitle

\pagenumbering{roman}

\section{Revision History}

\begin{tabularx}{\textwidth}{p{4cm}p{3cm}X}
\toprule {\bf Date} & {\bf Developer} & {\bf Notes}\\
\midrule
January 12th, 2025 & Kevin Ishak & Initialize template, add rough draft of section 3,4 and 5\\
January 13th, 2025 & Jianhao Wei & Add modules into section 6\\
January 13th, 2025 & Zain-Alabedeen Garada & Add modules into section 6\\
January 17th, 2025 & Zheng Bang Liang & Added all modules for Behavior Hiding Modules\\
January 17th, 2025 & Jianhao Wei, Kevin Ishak, Zain-Alabedeen Garada & Modify behavior hiding modules that Zheng added and add the rest of the MIS modules \\
January 17th, 2025 & Jianhao Wei & Modify section 3 and wrote section 4, 5, 16. Communicate with other members and wrote section 17\\
January 28th, 2025 & Jianhao Wei & Make changes based on peer feedbacks. Please see commits and issue trackers for detail\\
\bottomrule
\end{tabularx}

~\newpage

\section{Symbols, Abbreviations and Acronyms}

See MG document in \href{https://github.com/simon-0215/UNO-Flip-3D/blob/main/docs/Design/SoftArchitecture/MG.pdf}{here}.


\newpage

\tableofcontents

\newpage

\pagenumbering{arabic}

\section{Introduction}

UNO Flip is a modern twist on the traditional UNO card game, incorporating an innovative double-sided card deck with ”light” and ”dark” sides. Players are challenged to adapt their strategies dynamically as the game flips between these two modes. Our goal for this project is to design and develop a digital version of UNO Flip that emulates the physical gameplay experience while adding features like automated rule enforcement, multiplayer support, and interactive animations.

Complementary documents include the System Requirement Specifications and Module Guide. The full documentation and implementation can be found in \href{https://github.com/simon-0215/UNO-Flip-3D/tree/main}{here}. 


\section{Notation}



The structure of the MIS for modules comes from \citet{HoffmanAndStrooper1995},
with the addition that template modules have been adapted from
\cite{GhezziEtAl2003}.  The mathematical notation comes from Chapter 3 of
\citet{HoffmanAndStrooper1995}. Template used from the SFWRENG 4G06 GitHub in \href{https://github.com/smiths/capTemplate/tree/main/docs/Design/SoftDetailedDes}{here}. 

The following table summarizes the primitive data types used by \progname{UnoFlip3D}. 

\begin{center}
\renewcommand{\arraystretch}{1.2}
\noindent 
\begin{tabular}{l l p{7.5cm}} 
\toprule 
\textbf{Data Type} & \textbf{Notation} & \textbf{Description}\\ 
\midrule
Boolean & boolean & A variable that represent true or false on a statement \\
Integer number & int & A number without fractional part with the range between $[-2^{63}, 2^{63}-1]$\\
Decimal number & float & A number with fractional part represented by 32-bit single-precision float point\\
Data Stream & serializedData & A stream of binary data for inter-module or inter-device transmissions\\
\bottomrule
\end{tabular} 
\end{center}

The following table summarizes the derived object data types used by \progname{UnoFlip3D}

\begin{center}
\renewcommand{\arraystretch}{1.2}
\noindent 
\begin{tabular}{l l p{7.5cm}} 
\toprule 
\textbf{Object Type} & \textbf{Notation} & \textbf{Description}\\ 
\midrule
String Object & String & an object with a sequence of unicode characters that represent word or sentences\\
Generic Array Object & Array[Type] & A object represented by a set of certain type of variable \\
Dictionary Object & dictionary & A collection of two arrays with each variable in one array correspond to a specific element in another array\\
Graphics Description Object & GraphicObject & A object contain all the details that the user interface needed for display the game properly\\
\bottomrule
\end{tabular} 
\end{center}



\noindent
\progname{UnoFlip3D} uses functions, which
are defined by the data types of their inputs and outputs. Local functions are
described by giving their type signature followed by their specification.

\section{Module Decomposition}

The following description is taken directly from the Module Guide document for this project. The modules are divided into 3 main categories: Hardware-Hiding, Behaviour-Hiding and Software Decision. Below is a detailed description about how each modules is categorized:

\subsection{Hardwire-Hiding Modules}

\begin{itemize}
\item \hyperref[BSM]{\textbf{Backend/Server Module:}} Serves as a virtual hardware used by the rest of the system. This module provides the interface between the hardware and the software, allowing the system to display
outputs or accept inputs.
\end{itemize}

\subsection{Behaviour-Hiding Modules:} 

\begin{itemize}
\item \hyperref[CEM]{\textbf{Card Effect Module:}} Executes the effects of special cards and updates the game state accordingly
\item \hyperref[TMM]{\textbf{Turn Management Module:}} Manages the order of player turns, including handling special conditions like ”Reverse” or ”Skip” cards.
\item \hyperref[UIM]{\textbf{User Interface Module:}} Displays the game state to the user and accepts user inputs through various interactive elements.
\item \hyperref[SLM]{\textbf{Save/Load Module:}} Allows saving the current game state and loading it at a later time.
\item \hyperref[AM]{\textbf{Animation Module:}} Provides animations for card movements, flips, and game interactions.
\item \hyperref[OM]{\textbf{Output Module:}} Provides visual or textual outputs to the user based on the game state
\end{itemize}

\subsection{Software Decision Modules}

\begin{itemize}
\item \hyperref[MNM]{\textbf{Multiplayer Networking Module:}} Handles communication between players, including matchmaking, game state synchronization, and latency management.
\item \hyperref[VOM]{\textbf{Verification Output Module:}} Validates the final output of the game, ensuring compliance with rules and expected results.
\item \hyperref[IM]{\textbf{Input Module:}} Converts the input data into the data structure used by other modules, such as the game logic or UI modules.
\end{itemize}


\section{MIS of Backend/Server Module} 
\label{BSM}

\subsection{Module}
\hspace{1.5em}Backend/Server

\subsection{Uses}
\hspace{1.5em}None

\subsection{Syntax}

\subsubsection{Exported Constants}
\begin{itemize}
\item SUPPORTED\_DEVICES: Enumerates the supported hardware devices

\item DEFAULT\_RESOLUTION: Specifies the default screen resolution for the game.
\end{itemize}

\subsubsection{Exported Access Programs}

\begin{itemize}
\item initializeHardware()
\item captureInput()
\item renderOutput(graphicsData: GraphicsObject)
\item detectHardware()
\end{itemize}

\subsection{Semantics}

\subsubsection{State Variables}
\begin{itemize}
\item connectedDevices: Tracks the list of currently connected input/output devices.
\item currentResolution: Stores the current screen resolution of the application.
\end{itemize}

\subsubsection{Environment Variables}
\begin{itemize}
\item hardwareDrivers: Represents the drivers required to interface with the supported hardware.
\item platform: Indicates the operating system or platform the game is running on 

\end{itemize}

\subsubsection{Assumptions}
\begin{itemize}
\item All required hardware drivers are installed and operational.
\item The platform supports Unity's hardware abstraction layer.
\end{itemize}

\subsubsection{Access Routine Semantics}

\begin{itemize}
\item initializeHardware() $\rightarrow$ void\\
\textbf{Transition:} Sets up the required hardware connections and initializes drivers.

\item captureInput() $\rightarrow$ String\\
\textbf{Output:} Returns a structured object representing raw input data from connected devices.


\item renderOutput(graphicsData: GraphicsObject) $\rightarrow$ void\\
\textbf{Transition:} Translates graphical data into visual outputs using the rendering hardware.

\item detectHardware() $\rightarrow$ Array[String]\\
\textbf{Output:} Returns a list of hardware devices detected and compatible with the game.

\end{itemize}


\subsubsection{Local Functions}
\begin{itemize}
\item verifyDeviceSupport() $\rightarrow$ boolean\\
\textbf{Description:} Checks if the provided device is supported by the game.

\item applyDriverSettings() $\rightarrow$ void\\
\textbf{Description:} Configures hardware drivers based on the detected platform.

\item fallbackToDefault() $\rightarrow$ void\\
\textbf{Description:} Reverts to default hardware settings if the required device is not detected or initialization fails.
\end{itemize}







\section{MIS of Card Effect Module} 
\label{CEM}

\subsection{Module}
\hspace{1.5em}Card Effect

\subsection{Uses}
\hspace{1.5em}Hardwire Hiding

\subsection{Syntax}

\subsubsection{Exported Constants}
\begin{itemize}
\item DRAW\_TWO\_EFFECT: Specifies the effect identifier for a "Draw Two" card.
\item SKIP\_TURN\_EFFECT: Specifies the effect identifier for a "Skip" card.
\item FLIP\_DECK\_EFFECT: Specifies the effect identifier for a "Flip" card.
\end{itemize}

\subsubsection{Exported Access Programs}

\begin{itemize}
\item reverseDirection(playerId: int)
\item skipTurn(playerId: int)
\item triggerDrawCards(playerId: int, cardCount: Int)
\item flipDeck()
\end{itemize}

\subsection{Semantics}

\subsubsection{State Variables}
\begin{itemize}
\item currentEffect: Track the current effect being applied
\item effectQueue: Store the effects that are waiting to be applied
\end{itemize}

\subsubsection{Environment Variables}
\begin{itemize}
    \item The special card type that current game environment allowed.
\end{itemize}

\subsubsection{Assumptions}
\begin{itemize}
\item All card effects are predefined.
\item The "Flip" card effect toggles the entire game state between "light" and "dark" sides.
\end{itemize}

\subsubsection{Access Routine Semantics}

\begin{itemize}
\item reverseDirection(playerId: int, previousPlayerId: int): $\rightarrow$ void\\
\textbf{Transition:} Reverse the direction the game is being played to the previous player by re-assigning the previous player next opportunity

\item skipTurn(playerId: int, nextPlayerId: int) $\rightarrow$ void\\
\textbf{Transition:} Skip the opportunity for the specified player to play and assignment the opportunity to the next player

\item triggerDrawCards(playerId: int, cardCount: Int) $\rightarrow$ void\\
\textbf{Transition:} Let the player specified to draw another card into their database and add the card count to their totalPower variable.

\item flipDeck() $\rightarrow$ void\\
\textbf{Transition:} Changes the deckSide variable and updates the game state to reflect the flipped deck.

\end{itemize}


\subsubsection{Local Functions}
\begin{itemize}
\item calculateNextPlayer(direction: String) $\rightarrow$ int\\
\textbf{Description:} Determines the next player in the turn sequence after applying a "Skip" or "Reverse" effect.

\item applyChainEffect(effectQueue: Array[String]) $\rightarrow$ void\\
\textbf{Description:} Resolves multiple card effects in sequence defined in the input string array

\item toggleDeckSide() $\rightarrow$ void\\
\textbf{Description:} Switches the game state between "light" and "dark" sides during a "Flip" card effect.
\end{itemize}



\section{MIS of Turn Management Module} 
\label{TMM}

\subsection{Module}
\hspace{1.5em}Turn Management

\subsection{Uses}
\hspace{1.5em}Input, Card Effect

\subsection{Syntax}

\subsubsection{Exported Constants}
\hspace{1.5em}None

\subsubsection{Exported Access Programs}

\begin{itemize}
\item validateMove(playerId: int, cardid: int)
\item endTurn(playerId: int)
\item shuffleDeck()
\item drawCard(playerId: int)
\end{itemize}

\subsection{Semantics}

\subsubsection{State Variables}
\begin{itemize}
\item currentPlayer: Tracks the player whose turn it is
\item deck: Represents the stack of remaining cards in the game.
\item discardPile: Stores played cards
\item playerHands: Stores each player's cards
\end{itemize}

\subsubsection{Environment Variables}
\begin{itemize}
\item maxPlayers: Maximum number of players allowed in a game
\item flipEnabled: Boolean to toggle the flip functionality
\end{itemize}

\subsubsection{Assumptions}
\begin{itemize}
\item The number of players, game rules, player restrictions are preloaded
\item The game environment is known
\end{itemize}

\subsubsection{Access Routine Semantics}

\begin{itemize}
\item validateMove(playerId: int, cardId: int) $\rightarrow$ boolean\\
\textbf{Output:} Checks if a move is valid

\item endTurn(playerId:int) $\rightarrow$ void\\
\textbf{Transition:} Ends the current player's turn and starts the next

\item shuffleDeck() $\rightarrow$ void\\
\textbf{Transition:} Randomizes the card deck

\item drawCard(playerId: int) $\rightarrow$ void\\
\textbf{Transition:} Adds a card to the specified player’s hand

\end{itemize}


\subsubsection{Local Functions}
\begin{itemize}
\item shuffleProcess(original: Array[String]) $\rightarrow$ Array[String]\\
\textbf{Description:} Contain the random algorithm to shuffle the deck
\item CardModifier(cardId: int) $\rightarrow$ void\\
\textbf{Description:} Contain algorithm to draw different card to screen
\end{itemize}


\section{MIS of User Interface Module} 
\label{UIM}

\subsection{Module}
\hspace{1.5em}User Interface

\subsection{Uses}
\hspace{1.5em}Output, Turn Management

\subsection{Syntax}

\subsubsection{Exported Constants}
\begin{itemize}
\item DEFAULT\_THEME: Specifies the default theme for the game UI (e.g., light mode).
\item FONT\_STYLE: Default font style used across UI elements.
\item ASSET\_PATH: Directory path where assets are stored
\item DEFAULT\_CARD\_SPRITE: Specifies the default card sprite to use if none is provided.

\end{itemize}

\subsubsection{Exported Access Programs}

\begin{itemize}
\item updateCardDisplay(playerId: int, cardId: int)
\item showTurnIndicator(playerId: int)
\item displayMessage(message: String)
\item loadScene(type: String)
\item loadAsset(assetName: String)
\item unloadAsset(assetName: String)
\item playSound(effectName: String)
\end{itemize}

\subsection{Semantics}

\subsubsection{State Variables}
\begin{itemize}
\item displayedCards: Tracks the cards currently visible for each player.
\item turnIndicator: Indicates which player’s turn it is.
\item messageQueue: Stores pending notifications or chat messages to be displayed.
\item theme: Specifies the current visual theme in light mode or dark mode.
\item loadedAssets: Tracks assets currently loaded into memory.
\item audioSettings: Stores configuration for playing audio
\item assetCache: Cache for frequently accessed assets to improve performance.
\item assetDirectory: Path to the directory containing all assets 

\end{itemize}

\subsubsection{Environment Variables}
\begin{itemize}
\item The resolution of the device being used.
\end{itemize}

\subsubsection{Assumptions}
\begin{itemize}
\item The UI module assumes that game state updates from the multiplayer networking and turn management modules are reliable.
\item All required assets are preloaded by the Save/Load module.
\item Multiplayer synchronization ensures accurate real-time updates across all connected devices.
\item All assets are correctly named and stored in the specified directory.
\item The module assumes sufficient memory and storage are available for caching assets.
\item Dependencies for visual and audio formats are preinstalled on the system.
\end{itemize}

\subsubsection{Access Routine Semantics}

\begin{itemize}
\item updateCardDisplay(playerId: int, cardId: int) $\rightarrow$ void\\
\textbf{Transition:} Updates the player’s visible hand to reflect the current state of their cards.

\item showTurnIndicator(playerId: int) $\rightarrow$ void\\
\textbf{Transition:} Highlights the current player's turn using visual indicators.

\item displayMessage(message: String) $\rightarrow$ void\\
\textbf{Transition:} Displays a notification or chat message on the game screen.

\item loadScene(type: String) $\rightarrow$ void\\
\textbf{Transition:} Load specific type of background with animation to the user interface

\item loadAsset(assetName: String) $\rightarrow$ void\\
\textbf{Transition:} Loads the specified asset from the asset directory into memory and returns a reference.

\item unloadAsset(assetName: String) $\rightarrow$ void\\
\textbf{Transition:} Removes the specified asset from memory to free up resources.

\item playSound(effectName: String) $\rightarrow$ void\\
\textbf{Transition:} Plays the specified sound effect from the audio assets directory.

\end{itemize}


\subsubsection{Local Functions}
\begin{itemize}
\item applyTheme(themeId: int) $\rightarrow$ void\\
\textbf{Description:} Configures and applies the selected theme for the game UI.

\item renderMessageQueue(messages: Array[String]) $\rightarrow$ void\\
\textbf{Description:} Processes and displays pending messages in the queue.

\item adjustUILayout() $\rightarrow$ void\\
\textbf{Description:} Dynamically adjusts the layout based on the screen resolution and device type.

\item cacheAsset(assetName: String) $\rightarrow$ void\\
\textbf{Description:} Adds the specified asset to the cache for quick retrieval.

\item clearCache() $\rightarrow$ void\\
\textbf{Description:} Clears the asset cache to free up memory

\item validateAsset(assetName: String) $\rightarrow$ void\\
\textbf{Description:} Checks if the specified asset exists and is accessible.

\end{itemize}





\section{MIS of Save/Load Module} 
\label{SLM}

\subsection{Module}
\hspace{1.5em}Save/Load

\subsection{Uses}
\hspace{1.5em}Hardwire Hiding

\subsection{Syntax}

\subsubsection{Exported Constants}
\hspace{1.5em}None

\subsubsection{Exported Access Programs}

\begin{itemize}
\item save(info: String, description: String)
\item retrieve(description: String)
\item delete(description: String)
\item changeDesc(originalDesc: String, updateDesc: String)
\end{itemize}

\subsection{Semantics}

\subsubsection{State Variables}
\begin{itemize}
\item ifFull: Track if the database is full
\item dict: The dictionary that stores the array index correspond with descriptions
\item infoArray: The array that stores all the information
\end{itemize}

\subsubsection{Environment Variables}
\hspace{1.5em}None

\subsubsection{Assumptions}
\hspace{1.5em}The string and description stored does not contain any special characters

\subsubsection{Access Routine Semantics}

\begin{itemize}
\item save(info: String, description: String) $\rightarrow$ void\\
\textbf{Transition:} Save the information into the database with description

\item retrieve(description: String) $\rightarrow$ String\\
\textbf{Output:} Return the information by its description

\item delete(description: String) $\rightarrow$ void\\
\textbf{Transition:} Delete the information in the database by its description

\item changeDesc(originalDesc: String, updateDesc: String) $\rightarrow$ void\\
\textbf{Transition:} change the description of a piece of information into another

\end{itemize}


\subsubsection{Local Functions}
\begin{itemize}
\item returnIndex(description: String) $\rightarrow$ int\\
\textbf{Description:} Return the index of the infoArray based on the description. 
\end{itemize}





\section{MIS of Animation Module} 
\label{AM}

\subsection{Module}
\hspace{1.5em}Animation

\subsection{Uses}
\hspace{1.5em}User Interface, Card Effect, Save/Load

\subsection{Syntax}

\subsubsection{Exported Constants}
\hspace{1.5em}None

\subsubsection{Exported Access Programs}

\begin{itemize}
\item move(cardId: int, distance: int, direction: String)
\item flip(cardId: int)
\item select(cardId: int)
\item appear(cardId: int)
\item disappear(cardId: int)
\end{itemize}

\subsection{Semantics}

\subsubsection{State Variables}
\begin{itemize}
\item cardSide: Track side the card is on
\item cardColor: Track the color of the card
\item cardPosition: Track the position of the card 
\item show: Track if the card is shown on the screen
\end{itemize}

\subsubsection{Environment Variables}
\hspace{1.5em}None

\subsubsection{Assumptions}
\hspace{1.5em}Each card has a unique id


\subsubsection{Access Routine Semantics}

\begin{itemize}
\item move(cardId: int, distance: int, direction: String) $\rightarrow$ void\\
\textbf{Transition:} Move the card with specific id by a set amount of pixels with horizontal or vertical direction

\item flip(cardId: int) $\rightarrow$ void\\
\textbf{Transition:} Flip the card with specific id to show the opposite face

\item select(cardId: int) $\rightarrow$ void\\
\textbf{Transition:} Show the animation when the card is selected by the user

\item appear(cardId: int) $\rightarrow$ void\\
\textbf{Transition:} Show the card with specific id to the user screen

\item disappear(cardId: int) $\rightarrow$ void\\
\textbf{Transition:} Make the card with specific id to disappear from the user screen
\end{itemize}


\subsubsection{Local Functions}
\begin{itemize}
\item getCardInfo(id: int) $\rightarrow$ void\\
\textbf{Description:} Get the info of the card to local state variables
\item applyVisualElements(id: int) $\rightarrow$ void\\
\textbf{Description:} Apply the visual effect to the user screen based on the id provided and update local state variables
\end{itemize}



\section{MIS of Output Module} 
\label{OM}

\subsection{Module}
\hspace{1.5em}Output

\subsection{Uses}
\hspace{1.5em}Card Effect

\subsection{Syntax}

\subsubsection{Exported Constants}
\hspace{1.5em}None

\subsubsection{Exported Access Programs}

\begin{itemize}
\item render(info: String, font: int, color: String, location: int)
\item showCardEffect(id: int, effectNum: int)
\end{itemize}

\subsection{Semantics}

\subsubsection{State Variables}
\hspace{1.5em}None

\subsubsection{Environment Variables}
\hspace{1.5em}None

\subsubsection{Assumptions}
\hspace{1.5em}Each card has a unique id


\subsubsection{Access Routine Semantics}

\begin{itemize}
\item render(info: String, font: int, color: String, location: int)
$\rightarrow$ void\\
\textbf{Transition:} Display the information onto the screen with the font, color and location specified

\item showCardEffect(id: int, effectNum: int) $\rightarrow$ void\\
\textbf{Transition:} Using Card Effect module to show flip, skip or draw two on specific card 	

\end{itemize}


\subsubsection{Local Functions}
\begin{itemize}
\item checkEdge(font: int, location: int) $\rightarrow$ boolean\\
\textbf{Description:} Check if the information displayed exceeds the boundary of the screen
\end{itemize}


\section{MIS of Multiplayer Networking Module} 
\label{MNM}

\subsection{Module}
\hspace{1.5em}Multiplayer Networking

\subsection{Uses}
\hspace{1.5em}Verification Output, Save/Load, Animation

\subsection{Syntax}

\subsubsection{Exported Constants}
\hspace{1.5em}serverID: The serial number of the game room upon user request
\subsubsection{Exported Access Programs}
\begin{itemize}
\item createGameRoom(playerId: int, roomSettings: Array[String])
\item joinGameRoom(playerId: int, roomId: int)
\item broadcastUpdate(gameId: int, update: String)
\end{itemize}

\subsection{Semantics}

\subsubsection{State Variables}
\begin{itemize}
\item activeGames: Tracks all ongoing game sessions.
\item connectedPlayers: List of currently connected players.
\end{itemize}


\subsubsection{Environment Variables}
\begin{itemize}
\item serverIP: IP address of the game server.
\item timeoutLimit: Time limit for a player to respond during their turn.
\end{itemize}

\subsubsection{Assumptions}
\begin{itemize}
\item The connection between server and other machines can be established successfully
\item The encryption and decryption methods are known
\end{itemize}

\subsubsection{Access Routine Semantics}

\begin{itemize}
\item createGameRoom(playerId: int, roomSettings: Array[String]) $\rightarrow$ void\\
\textbf{Transition:} Creates a new game room by a specific user with specific setting


\item joinGameRoom(playerId: int, roomId: int, publicKey: int) $\rightarrow$ int\\
\textbf{Transition:} Adds a specific player to an existing room by its ID and public key for encryption and decryption purpose purposes \\
\textbf{Output:} Return the public key of the server for encryption and decryption purposes

\item broadcastUpdate(gameId: int, update: String) $\rightarrow$ void\\
\textbf{Transition:} Sends game state updates to all players in a room.

\end{itemize}


\subsubsection{Local Functions}
\begin{itemize}
\item encryption(information: String, publicKey: int) $\rightarrow$ String\\
\textbf{Description:} Contain encryption algorithm to encrypt data before sending using public key from user
\item decryption(information: String, privateKey: int) $\rightarrow$ String\\ 
\textbf{Description:} Contain decryption algorithm to decrypt data after receiving using the private key of game room
\end{itemize}


\section{MIS of Verification Output Module}
\label{VOM}

\subsection{Module}
\hspace{1.5em}Verification Output

\subsection{Uses}
\hspace{1.5em}None

\subsection{Syntax}

\subsubsection{Exported Constants}
\hspace{1.5em}None

\subsubsection{Exported Access Programs}

\begin{itemize}
\item captureOutput(playerId: int, info: String)
\item validateOutput(info: String)
\end{itemize}

\subsection{Semantics}

\subsubsection{State Variables}
\begin{itemize}
\item outputBuffer: Temporarily store the incoming input received for later use
\item validatedOutput: Store the input that has been validated by the module for later transmission
\end{itemize}

\subsubsection{Environment Variables}
\begin{itemize}
\item the validation algorithm the device is running on
\end{itemize}

\subsubsection{Assumptions}
\begin{itemize}
\item All output devices conform to Unity's input standard.
\item The validation algorithm must make sure that there is no error or discrepancy occurring after the validation
\end{itemize}

\subsubsection{Access Routine Semantics}

\begin{itemize}
\item captureInput(info: String) $\rightarrow$ void\\
\textbf{Transition:} Capture and save the information into the output buffer

\item validateInput(info: String) $\rightarrow$ String, boolean\\
\textbf{Transition:} validate the output from the outputBuffer using existing algorithms\\
\textbf{Output:} Return the original output and a boolean indicating if the input can be validated


\end{itemize}


\subsubsection{Local Functions}
\begin{itemize}
\item algorithmDatabase(input: String, type: int) $\rightarrow$ String\\
\textbf{Description:} Contain the algorithm that converts the input string to the format that can be used by other modules and return the converted input string

\item serialization(input: String) $\rightarrow$ serializedData\\
\textbf{Description:} Contain the algorithm to convert the input string into serialized data for inter-module or internet communications
\end{itemize}






\section{MIS of Input Module} 
\label{IM}

\subsection{Module}
\hspace{1.5em}Input

\subsection{Uses}
\hspace{1.5em}Hardwire Hiding

\subsection{Syntax}

\subsubsection{Exported Constants}
\hspace{1.5em}None

\subsubsection{Exported Access Programs}

\begin{itemize}
\item captureInput(playerId: int, info: String)
\item validateInput(info: String)
\item convertInput(info: String, type: String)
\end{itemize}

\subsection{Semantics}

\subsubsection{State Variables}
\begin{itemize}
\item inputBuffer: Temporarily store the incoming input received for later use
\item validatedInput: Store the input that has been validated by the module for later transmission
\end{itemize}

\subsubsection{Environment Variables}
\begin{itemize}
\item The version of supporting device that the software is running on
\item the validation algorithm the device is running on
\end{itemize}

\subsubsection{Assumptions}
\begin{itemize}
\item All input devices conform to Unity's input standard.
\item The validation algorithm must make sure that there is no error or discrepancy occurring after the validation
\end{itemize}

\subsubsection{Access Routine Semantics}

\begin{itemize}
\item captureInput(playerId: int, info: String) $\rightarrow$ void\\
\textbf{Transition:} Capture and save the information into the input buffer

\item validateInput(info: String) $\rightarrow$ String, boolean\\
\textbf{Transition:} validate the input from the inputBuffer using existing algorithms\\
\textbf{Output:} Return the original input and a boolean indicating if the input can be validated

\item convertInput(action: String, type: String) $\rightarrow$ String\\
\textbf{Output:} Convert the input from validatedInput into specific format that can be used by other modules

\end{itemize}


\subsubsection{Local Functions}
\begin{itemize}
\item algorithmDatabase(input: String, type: int) $\rightarrow$ String\\
\textbf{Description:} Contain the algorithm that converts the input string to the format that can be used by other modules and return the converted input string

\item serialization(input: String) $\rightarrow$ serializedData\\
\textbf{Description:} Contain the algorithm to convert the input string into serialized data for inter-module or internet communications
\end{itemize}


\section{Exception Handling Strategies}
The exception handling is critical for our software since it directly impacts the user experience of our software. It is our responsibility to ensure that our customers have a good experience with our software. To prevent exception from happening in our software, we implement the following 4 strategies:
\begin{itemize}
    \item \textbf{Limit Erroneous User Input:} We design the user interface such that the user input is bounded within a certain range to limit erroneous user input that might crash the software. We also include the input verification in Input module to ensure all the information that passed to the software are legitimate
    \item \textbf{Wrap External Resources:} We have design all of our function in our modules to wrap the resources and libraries they use  from the global space of the software. This ensures that the exception in third-party software does not impact the integrity of own software.
    \item \textbf{Cleaning up resources:} We have implemented the mechanism to clean up unused resource promptly and reliably to make sure the exceptions do not occurs due to cache overload
    \item \textbf{Limiting Errors Instead of Handling Errors:} Instead of designing exception handling mechanism, we make sure our software is carefully designed and tested to reduce the chance of exception happening.
\end{itemize}
By implementing these strategies, we can reduce the chance for exception happening and limit the need of the exception handling mechanisms.



\bibliographystyle {plainnat}
\bibliography {../../../refs/References}
\begin{enumerate}
    \item M. Xu, "UNO Flip 3D - SRS Volere Documentation," \textit{GitHub Repository}, 2023.
    \item M. Xu, "UNO Flip 3D - Software Architecture Document," \textit{GitHub Repository}, 2023.
\end{enumerate}

\newpage
\section{Appendix} \label{Appendix}





\section*{Appendix --- Reflection}



The information in this section will be used to evaluate the team members on the
graduate attribute of Problem Analysis and Design.



\begin{enumerate}
  \item What went well while writing this deliverable? \\ \\
  Our team is able to divide the task very fast, and everyone is working more collaboratively than before. We also started this deliverable earlier than before and be able to present the rough draft during the informal TA meeting. We get lots of feedback from our TA and we are confident that we can get a higher grade than before. In terms of documents, the module guide is relatively quick to do. We are able to gather information very fast and the communication went really well.
  
  \item What pain points did you experience during this deliverable, and how did you resolve them?\\ \\
  The pain in this deliverable is the implementation of MIS document. The function and variables in every module are hard to visualize because the relationship between the module are very complex and everyone have their own opinion about how the implementation. We have to have extended meeting session to discuss about concept and resolve the conflict between different team members. We tried to absorb the advantages from the opinions from different team member and coming to an integrated idea. But this take a lot of time.
  
  \item Which of your design decisions stemmed from speaking to your client(s) or a proxy (e.g. your peers, stakeholders, potential users)? For those that were not, why, and where did they come from?\\ \\
  The MIS are mainly coming from communicating with our peers and our supervisor. But for software architecture part of our decision, we simply reading the materials on the internet such as past project to get a better feeling about the most abstract part of our implementation.
  
  \item While creating the design doc, what parts of your other documents (e.g.requirements, hazard analysis, etc), it any, needed to be changed, and why?\\ \\
  The document requirements need to be change because we discovered new idea while implementing our project. There are some requirement in the original document doesn't fit with the software architecture we chose. Some of the requirement are also too vague to be implemented properly, some of the requirement are too hard to implement in the actual software. We have to change our SRS document according to the software architecture and modules we implemented, and remove the unnecessary and vague requirement to make our documentation more consistent and integrate. 
  
  \item What are the limitations of your solution?  Put another way, given
  unlimited resources, what could you do to make the project better? (LO\_ProbSolutions)\\ \\
  The limitations of our solution is that we never implemented similar software before and this is our first time experience with this kind of project. Even though we have internet but our view is still limited. If the resource and time is unlimited, we would consult with professional people (such as professionals from game companies) to get a better understanding about how should this project be implemented and what is the more efficient way to build this kind of project and managment teams.
  \item Give a brief overview of other design solutions you considered.  What are the benefits and tradeoffs of those other designs compared with the chosen design?  From all the potential options, why did you select the documented design? (LO\_Explores)\\ \\
  We had considered about design the same game using AI powered opponent with a single player. The AI solution will more cool and this solution will become more convenient with people who are lonely and don't have the access of the internet. But, the AI solution is also much more challenging and the chance of failure is much greater. The single player might also be less popular than multiplayer game for the public since multiplayer game are more fun and engaging. We select our current solution because it achieve the trade-off between implementation difficulties and public popularity.
\end{enumerate}


\end{document}