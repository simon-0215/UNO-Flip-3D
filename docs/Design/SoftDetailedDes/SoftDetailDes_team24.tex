\documentclass{article}
\usepackage{geometry}
\geometry{a4paper, margin=1in}
\usepackage{hyperref}
\usepackage{longtable}
\usepackage{amsmath}

\title{Module Interface Specification for UnoFlip3D}
\author{Team24 unomaster \\ Mingyang Xu, Kevin Ishak, Jianhao Wei, Zain-Alabedeen Garada, Zheng Beng Liang}
\date{January 15th, 2025}

\begin{document}

\maketitle

\tableofcontents

\newpage

\section{Revision History}
\begin{longtable}{|l|l|l|}
\hline
\textbf{Date} & \textbf{Version} & \textbf{Notes} \\
\hline
January 12th, 2025 & 0 & Initial Commit \\
\hline
\end{longtable}

\section{Symbols and Acronyms}
% Add symbols and acronyms here if available.

\section{Introduction}
The following document details the Module Interface Specifications for UNO Flip. \\ 
\textit{ADD PROJECT DESCRIPTION} \\ 
Complementary documents include the System Requirement Specifications and Module Guide. The full documentation and implementation can be found at: 
\url{https://github.com/simon-0215/UNO-Flip-3D/tree/main}

\section{Notation}
The structure of the MIS for modules comes from Hoffman and Strooper (1995), with adaptations from Ghezzi et al. (2003) for template modules. Mathematical notation is drawn from Chapter 3 of Hoffman and Strooper (1995). For example, \texttt{:=} denotes a multiple assignment statement, and conditional rules follow the form \((c_1 \Rightarrow r_1 | c_2 \Rightarrow r_2 | \dots | c_n \Rightarrow r_n)\).

\begin{longtable}{|l|l|p{8cm}|}
\hline
\textbf{Data Type} & \textbf{Notation} & \textbf{Description} \\
\hline
Character & \texttt{char} & A single symbol or digit \\
\hline
Integer & \(\mathbb{Z}\) & A number without a fractional component in \((-\infty, \infty)\) \\
\hline
Natural Number & \(\mathbb{N}\) & A number without a fractional component in \([1, \infty)\) \\
\hline
Real & \(\mathbb{R}\) & Any number in \((-\infty, \infty)\) \\
\hline
\end{longtable}

\section{Module Decomposition}
\begin{longtable}{|l|l|}
\hline
\textbf{Level 1} & \textbf{Level 2} \\
\hline
Hardware-hiding & Device Input \\ 
& Audio Output \\ 
& Screen Rendering \\
\hline
Behaviour-hiding & Game Rules Logic \\ 
& Turn Management \\ 
& Card Management \\ 
& Player Interaction \\
\hline
Software Decision & Networking Protocol \\ 
& Multiplayer System \\
\hline
\end{longtable}

\section{MIS of ``Module Name''}
\subsection{Module}
% Define the module here.

\subsection{Uses}
% Specify module dependencies here.

\subsection{Syntax}
\subsubsection{Exported Constants}
% List exported constants.

\subsubsection{Exported Access Programs}
% List exported access programs.

\subsection{Semantics}
\subsubsection{State Variables}
% Define state variables.

\subsubsection{Environment Variables}
% Define environment variables.

\subsubsection{Assumptions}
% List assumptions.

\subsubsection{Access Routine Semantics}
% Describe access routine semantics.

\subsubsection{Local Functions}
% List local functions.

\section{References}
% Add references here.

\end{document}